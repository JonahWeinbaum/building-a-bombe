\chapter{Permutations}

\begin{definition}
    A \textbf{permutation} $\sigma$ is a bijective function from a set $S$ to 
    the same set 
    \begin{center}
        $\sigma$: $S\to S$
    \end{center}
\end{definition}

A permutation can be thought of as ``swapping'' elements of our set while. Some simple examples 
of permutations are
\begin{example}
    \text{}
    \begin{enumerate}
        \item$\sigma = \text{id}_S$ known as the \textbf{identity permuation}, which maps each element to itself
        \item$S = \{a, b, c, d, \dots, z\}$ and $\sigma$ defined by $\sigma(a) = b, \dots, \sigma(z) = a$. This shifts 
        each letter by one place in the alphabet, wrapping around at $z$. We will called this the \textbf{Caesar permutation} and it will be denoted by 
        $\theta_1$
    \end{enumerate}
\end{example}

We will show that for a fixed rotor state the Enigma functions is a permutation on the set of the latin alphabet.
It is clear that the enigma machine sends this set to itself but it is not immediately apparent that such a mapping is bijective. To see this
we must shift our discription of the enigma to that of a composition of permutations. Let us first define some useful notation.

\begin{definition}
    For a set $S$, the \textbf{symmetric group over $S$} is the group $(G,\circ)$ of all permutations on $S$, where
    \begin{center}
        $G\coloneq\{f: S\to S\text{ }\vert\text{ }f\text{ is a bijection}\}$
    \end{center}
    where the group operation is composition of functions. We denote this group $\operatorname{Sym}(S)$.
\end{definition}

It is clear that this group is closed with respect to composition, since the composition of bijections is itself a bijection. The identity element is simply the identity permutation described in CITE, and 
the inverse of an element is simply its inverse as a function, since the inverse of a bijection is itself a bijection this is well-defined.

\begin{example}
    The Caeser cipher is one of the simplest and earliest encryption schemes. It involves shifting
    the set of letters by a fixed amount to encode a message. In Caesar's case he would shift each letter in a message by three places, 
    sending $A\mapsto D,\dots, X\mapsto A, Y\mapsto B, Z\mapsto C$. In the context of permuations, this can be viewed as a repeated
    application of the Caesar permuation $\theta_1$ described earlier CITE. For instance, to get Caesar's particular cipher 
    we use $\theta_1\circ\theta_1\circ\theta_1$ (that is $\theta_1^3\in\operatorname{Sym}(\{A,\dots,Z\})$). For ease of notation we define
    \begin{center}
        $\theta_n\coloneq\theta_1^n$ for $n\in\mathbb{N}$
    \end{center}
    Though this continues indefinitely, due to the nature of the permuation it is clear that $\theta_n = \theta_{(n+26)}$ $\forall\text{ }n\in\mathbb{N}$ so, in particular, 
    there are only $26$ distinct Caesar ciphers -- a small keyspace indeed.
\end{example}

\section{Enigma as a Permutation} 

Suppose for now that we have fixed the rotors of the enigma machine in place. We want to examine how each letter is mapped through the machine. 
Recall from CITE that the enigma machine maps each letter signal through the following transformations: the plugboard, three rotors, a reflector panel, back through the rotors, 
and back through the plugboard. Each of these components can themselves be viewed as a bijective mapping by construction. We will label the permutation associated with each  as follows:
\begin{enumerate}
    \item $R_{(1,r_1, \ell_1, g_1)}, R_{(2, r_2, \ell_2, g_2)}, R_{(3, r_3, \ell_3, g_3)}$ will be the rotors going from right to left where 
    $r_i\in \{\text{I}, \text{II}, \dots, \text{VI}\}$ define which rotors we have selected, and $\ell_i, g_i\in\{a, \dots, z\}$ indicates the \emph{Ringstellung} and \emph{Grundstellung} respectively for that rotor.
    \item $P$ will be our plugboard setting
    \item $M_\alpha$ will be our reflector setting where $\alpha\in\{A, B, C\}$ describes our reflector type
\end{enumerate}

For the rotor and the reflector we include subscripts to describe their settings since their permutation is entirely determined by this short list of parameters.
The plugboard, on the other hand, is itself (by the \emph{Steckerverbindungen}) a description of its permutation so subscripts are unnecessary.
\\\\Supposing we are at ground position, the engima permutation $E$ is as follows:
\begin{center}
    $E = PR_1R_2R_3M_\alpha R_3^{-1} R_2^{-1} R_1^{-1} P^{-1}$
\end{center}
\noindent We note that
\begin{align*}
    E &= PR_{(1,r_1, \ell_1, g_1)}R_{(2, r_2, \ell_2, g_2)}R_{(3, r_3, \ell_3, g_3)}M_\alpha R_{(3, r_3, \ell_3, g_3)}^{-1} R_{(2, r_2, \ell_2, g_2)}^{-1} R_{(1,r_1, \ell_1, g_1)}^{-1} P^{-1}
    \\&=(PR_{(1,r_1, \ell_1, g_1)}R_{(2, r_2, \ell_2, g_2)}R_{(3, r_3, \ell_3, g_3)})\circ M_\alpha \circ(PR_{(1,r_1, \ell_1, g_1)}R_{(2, r_2, \ell_2, g_2)}R_{(3, r_3, \ell_3, g_3)})^{-1}
\end{align*}