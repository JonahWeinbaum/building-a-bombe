\chapter{Conclusion}\label{conclusion}
\section{H-M Factor}
% https://wdc.contentdm.oclc.org/digital/collection/leo/id/167 %
This entire discussion of the expected number of stops has ignored
the impact of the diagonal board. Once implemented, the diagonal
board drastically reduced the number of stops for a run of the Bombe.
\subsection{Turing's H-M Factor}
To describe this effect, Turing introduced an additional term called
the {\bf{H-M factor}}, named after Cyril Holland-Martin the Technical
Director of the British Tabulating Company which manufactured the Bombe~\cite[p.~116]{Turing1940ProfBook}.
\\\\The H-M factor measures the proportion of normal stops over all
steckering hypotheses that are maintained after introducing the
diagonal connections. Given that the number of diagonal wires being
used is dictated by the number of letters being considered in a menu,
Turing gave separate H-M factors for menus containing various numbers
of letters. We will denote the H-M factor for a menu with $\ell$ letters
as $\mathcal{H}_\ell$.
\begin{figure}[H]
\centering
\scalebox{0.9}{
\begin{tabular}{|c|c|}
\hline
\textbf{Distinct Letters in Menu} & \ \ \ \ \ \ \ \ \ \ \ \ $\boldsymbol{\mathcal{H}}_\ell$\ \ \ \ \ \ \ \ \ \ \ \ \\
\hline
2 & 0.92 \\
\hline
3 & 0.79 \\
\hline
4 & 0.62 \\
\hline
5 & 0.44 \\
\hline
6 & 0.29 \\
\hline
7 & 0.17 \\
\hline
8 & 0.087 \\
\hline
9 & 0.041 \\
\hline
10 & 0.016 \\
\hline
11 & 0.0060 \\
\hline
12 & 0.0018\\
\hline
13 & 0.00045\\
\hline
14 & 0.000095\\
\hline
15 & 0.000016\\
\hline
16 & 0.0000023\\
\hline
\end{tabular}}
\caption{The H-M factor as given by Turing in the \emph{Prof's Book}~\cite[p.~116]{Turing1940ProfBook}}
\end{figure}
\noindent Supposing we had a menu with $c$ closures containing $\ell$ distinct
letters. Turing's model tells us that before the introduction of the
diagonal board we expect $26^{4-c}$ normal stops over all steckering
hypotheses. Given that $\mathcal{H}_\ell$ of these normal stops are
maintain after introducing the diagonal board, Turing's model
estimates that we expect
\[
  26^{4-c}\cdot\mathcal{H}_\ell
\]
total normal stops over all steckering hypotheses when including the
diagonal board.
\\\\The H-M factor has been discussed in academic literature
regarding the Bombe, with researchers like John Wright~\cite{Wright2015} and Donald W. Davies~\cite{Davies2010} attempting to improve the accuracy of the H-M factor through
either recursive formulae or explicit enumeration, respectively. Much
of this research stems from the fact that Turing provides no formal
justification for his computation of the H-M factor, stating simply
that, ``the method of constructing the [H-M factor] table is very
tedious and uninteresting''~\cite[p.~116]{Turing1940ProfBook}. This understatement sparked a number of
attempts to recreate, or improve upon, this ``uninteresting'' computation.
\\\\For the same reasons discussed in Section~\ref{justify_turing},
this serves as a fair approximation of the actual number of stops
since with more loops the proportion of stops with type $1^125^1$
increases. However, this also means that the calculation of
$\mathcal{H}_\ell$ suffers from the inherent issues with Turing's model
without the diagonal board -- under-counting due to abnormal stops,
and over-counting stops with multiple steckering hypotheses producing
normal stops. As an example, Turing calculates that $\mathcal{H}_2 =
0.92$~\cite[p.~116]{Turing1940ProfBook}. While this may serve as an accurate estimate of the number of
normal stops over all steckering hypotheses, it has scarce relation
to the actual number of stops. For instance, a menu with one closure
and only two letters can never produce a stop -- even with the
introduction of the diagonal board.
\subsection{Cycle-Type Based H-M Factor}\label{cyclehm}
As with our cycle-type model introduced in the last section, a proper
calculation of $\mathcal{H}_\ell$ should not only take into account the
single variable $\ell$, it should also take into account an additional
variable representing the partition type $\upsilon\in\mathcal{I}_n$ of a stop.
Further, such an H-M factor would not give the proportion of normal
stops over all steckering hypotheses that are maintained after the
introduction of the diagonal board, it would give the proportion of
\emph{all} stops which are maintained. We will denote such a
hypothetical H-M factor as $\mathcal{H}_{\ell,\upsilon}$. While
$\mathcal{H}_\ell$ is Turing's original scalar correction factor for
menus of length $\ell$, our refinement $\mathcal{H}_{\ell,\upsilon}$
introduces partition-aware correction accounting for the stop type of a stop.
\\\\Consider the case of two closures. We showed in the last section
that we can compute the probability of a stop for two loops with
cycle structures $\alpha$ and $\beta$ as.
\begin{align*}
  t_{\alpha, \beta}\ = 1 -
  \frac{Z(\alpha)Z(\beta)}{(n!)^2}\sum_{(\upsilon_1,\dots,\upsilon_k)\ne(n)}\frac{n!}{Y(m^{(\upsilon)})}\sum_{\alpha^{(\upsilon_j)}\in
  S_{\alpha,\upsilon}}\sum_{\beta^{(\upsilon_j)}\in
  S_{\beta,\upsilon}}\prod_{j=1}^k
  {\frac{(\upsilon_j!)^2}{Z(\alpha^{(\upsilon_j)})Z(\beta^{(\upsilon_j)})}}\cdot
  t_{\alpha^{(\upsilon_j)},\beta^{(\upsilon_j)}}.
\end{align*}
We can modify our version of Dixon's Theorem to give us the
likelihood that two permutations $\sigma,\tau\in S_n$ with cycle type
$\alpha$ and $\beta$ are such that the orbits of
$\langle\sigma,\tau\rangle$ have a particular partition type
$\upsilon\in\mathcal{I}_n$. We
denote this probability $t_{\alpha,\beta}^{(\upsilon)}$ and this is given as
\begin{align*}
  t_{\alpha,\beta}^{(\upsilon)}\ =\frac{Z(\alpha)Z(\beta)}{(n!)^2}\frac{n!}{Y(m^{(\upsilon)})}\sum_{\alpha^{(\upsilon_j)}\in
  S_{\alpha,\upsilon}}\sum_{\beta^{(\upsilon_j)}\in
  S_{\beta,\upsilon}}\prod_{j=1}^k
  {\frac{(\upsilon_j!)^2}{Z(\alpha^{(\upsilon_j)})Z(\beta^{(\upsilon_j)})}}\cdot
  t_{\alpha^{(\upsilon_j)},\beta^{(\upsilon_j)}}.
\end{align*}
We note that
\[
  t_{\alpha,\beta} = 1 -
  \sum_{\upsilon\ne(n)}t_{\alpha,\beta}^{(\upsilon)}.
\]
Recall that the probability of a stop for loops of length $a$ and $b$ is
\begin{align*}
  &1-\sum_{\alpha,
  \beta}\mathbb{P}(\overline\pi_1\dots\overline\pi_a\text{ has cycle
      type
  }\alpha)\cdot\mathbb{P}(\overline\delta_1\dots\overline\delta_b\text{
  has cycle type }\beta)\cdot t_{\alpha, \beta}\\
  = \text{ }&1-\sum_{\alpha,
  \beta}\mathbb{P}(\overline\pi_1\dots\overline\pi_a\text{ has cycle
      type
  }\alpha)\cdot\mathbb{P}(\overline\delta_1\dots\overline\delta_b\text{
  has cycle type }\beta)\cdot (1 -
  \sum_{\upsilon\ne(n)}t_{\alpha,\beta}^{(\upsilon)})\\
  = \text{ }&1-\sum_{\alpha,
  \beta}\mathbb{P}(\overline\pi_1\dots\overline\pi_a\text{ has cycle
      type
  }\alpha)\cdot\mathbb{P}(\overline\delta_1\dots\overline\delta_b\text{
  has cycle type }\beta)
  \\\text{ }&+
  \sum_{\alpha,\beta}\sum_{\upsilon\ne(n)}\mathbb{P}(\overline\pi_1\dots\overline\pi_a\text{
      has cycle type
  }\alpha)\cdot\mathbb{P}(\overline\delta_1\dots\overline\delta_b\text{
  has cycle type }\beta)\cdot t_{\alpha,\beta}^{(\upsilon)}\\
  =\text{
  }&\sum_{\alpha,\beta}\sum_{\upsilon\ne(n)}\mathbb{P}(\overline\pi_1\dots\overline\pi_a\text{
      has cycle type
  }\alpha)\cdot\mathbb{P}(\overline\delta_1\dots\overline\delta_b\text{
  has cycle type }\beta)\cdot t_{\alpha,\beta}^{(\upsilon)}\\
  =\text{
  }&\sum_{\upsilon\ne(n)}\sum_{\alpha,\beta}\mathbb{P}(\overline\pi_1\dots\overline\pi_a\text{
      has cycle type
  }\alpha)\cdot\mathbb{P}(\overline\delta_1\dots\overline\delta_b\text{
  has cycle type }\beta)\cdot t_{\alpha,\beta}^{(\upsilon)}
\end{align*}
This does not include our hypothetical H-M factor yet. Our factor
$\mathcal{H}_{\ell,\upsilon}$, gives the proportion of stops of type
$\upsilon$ that
are maintained after the diagonal board is introduced. Then the total
number of expected stops, with two closures with lengths $a$ and $b$
representing a menu of $\ell$ distinct letters ($\ell$ is not necessarily
$a+b$ because the closures may share letters), including the diagonal board is,
\[
  \sum_{\upsilon\ne(n)}\mathcal{H}_{\ell,\upsilon}\sum_{\alpha,\beta}\mathbb{P}(\overline\pi_1\dots\overline\pi_a\text{
      has cycle type
  }\alpha)\cdot\mathbb{P}(\overline\delta_1\dots\overline\delta_b\text{
  has cycle type }\beta)\cdot t_{\alpha,\beta}^{(\upsilon)}.
\]
While this provides a framework for
understanding such a hypothetical factor $\mathcal{H}_{\ell,\upsilon}$ -- we
leave its explicit computation to future work.
\\\\In theory, $\mathcal{H}_{\ell,\upsilon}$ could be estimated via simulation;
However, this would require that we are able to efficiently generate
menus producing arbitrary orbit structures, many of which are
extremely rare. This would likely result in high variance and poor convergence.
\\\\We suspect it is possible to perform an exact enumeration to
compute $\mathcal{H}_{\ell,\upsilon}$, likely via similar methods used to
compute $t_{\alpha,\beta}$, though we do not yet posses such a
procedure. Developing a means to compute this factor would allow us
to model the effective number of stops with high accuracy across
nearly all Bombe configurations — with or without the diagonal board.

\section{Computational Methods}
Many of the results in this thesis were obtained through simulations
and computational analysis. To support reproducibility and encourage
further exploration, this thesis is accompanied by a companion
repository containing all relevant code.
This repository can be found at
\url{https://github.com/JonahWeinbaum/building-a-bombe}. This
repository includes the scripts used for:
\begin{itemize}
  \item Simulating the Bombe
  \item Simulating the Enigma
  \item Creating and using Zygalski Sheets
  \item Computing scoring, distance, dummyismus, and repeat sheets
    for Banburismus
  \item Capturing cycle distributions in the Bombe
  \item Computing probabilities $t_{\alpha,\beta}$
  \item Computing the expected number of stops in the Bombe for
    various menu arrangements
\end{itemize}
This repository is intended not only to allow readers to experiment
with the methods described herein, but also to serve as a foundation
for any future research building on this work.

\section{Future Work} 
This thesis provides several promising directions for
exploration. We hope researchers can improve the estimation of stops in the Bombe by
computing a cycle type based H-M factor as in Section~\ref{cyclehm}. Further, we hope readers find novel use cases for the generalized Dixon's Theorem~\ref{general_dixon}. This may serve to deepen investigations of transitivity
in randomly composed permutations or, via reduction of the theorem statement, provide new results in other fields. In particular, we have begun investigating a relationship between the probability that a graph pulled from a uniform distribution over bipartite graphs is connected and the statement of the generalized Dixon's Theorem. Finally, through further optimization of the algorithm implementing the generalized Dixon's Theorem~\ref{general_dixon}, researchers will be able to investigate the ways in which additional closures in a menu affect the number of stops the Bombe is expected to encounter.
\section{Conclusion}
Throughout this thesis, we have described a battle of mathematics, a cat and mouse game between Allied cryptanalysts and Axis cryptographers attempting to thwart each other with a variety of cryptographic vulnerabilities and countermeasures. Our contribution to this story was to amalgamate and expand on the mathematical reasoning underlying this battle. We provide uniquely comprehensive and consistent mathematical rigor meant to inform modern audiences about a key period in the history of computing and cryptography. 
\\\\At the heart of this thesis is the belief that exploring subjects that may be considered dated or antiquated can still yield novel and modern insights. Learning about the problems that necessitated the creation of modern computing and understanding the thought processes of those who solved those problems can illuminate new perspectives and raise new questions in the field of computing. Examining this story provided us with a unique problem whose solution, in the form of the generalized Dixon's Theorem, offers a novel contribution to modern research in the field of abstract algebra. In analyzing and extending
Turing’s approach with modern tools, we not only gain greater understanding
of the Bombe itself, but also of the mathematical structures that
underlie its success. 
\\\\This thesis serves to bridge a gap between historical cryptanalysis and contemporary mathematics. The stories we have examined are not relics of the past, but living lessons in perseverance and intellect which can shape our understanding of modern cryptanalysis and mathematics. The mathematical reasonings themselves are not bound by the time in which they were created; in contemporary reexamination, we find new avenues for exploration and discovery. 
