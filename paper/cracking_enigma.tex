
\documentclass{dcthesis}

\usepackage[export]{adjustbox} % in preambl
\usepackage[normalem]{ulem}
\usepackage{circuitikz}
\usepackage{longtable}
\usepackage{subcaption}
\usepackage{graphicx}
\usepackage{setspace}
\usepackage{amssymb,amsthm,amsmath,amstext}
\usepackage{mathrsfs}  % allows nice script font for sheaves
% \usepackage{bm}        % allows bold italic letters in math mode
% \usepackage{mathtools} % allows more extendible arrows
\usepackage{stmaryrd}  % additional nice fonts
\usepackage{mathtools}
\usepackage{geometry}
\usepackage{xy}
\usepackage{algorithm}
\usepackage{algpseudocode}
\usepackage{lipsum}
\usepackage{caption}
\usepackage{circuitikz}
\usepackage{tikz-cd}
\usepackage{bbm}
\usepackage{hyperref}
\usepackage{enumerate}% http://ctan.org/pkg/enumerate
\usepackage{tikz}
\usepackage{tikz-cd}
\usepackage{dirtytalk}
\usetikzlibrary{calc}
\usetikzlibrary{positioning, shapes, shapes.geometric, fit}
\hypersetup{
  colorlinks=true,
  linkcolor=blue,
  filecolor=magenta,
  urlcolor=cyan,
  pdftitle={Overleaf Example},
  pdfpagemode=FullScreen,
}

\makeatletter
\renewcommand{\@chapapp}{}% Not necessary...
\newenvironment{chapquote}[2][2em]
{\setlength{\@tempdima}{#1}%
  \def\chapquote@author{#2}%
  \parshape 1 \@tempdima \dimexpr\textwidth-2\@tempdima\relax%
\itshape}
{\par\normalfont\hfill--\ \chapquote@author\hspace*{\@tempdima}\par\bigskip}
\makeatother

\DeclareCaptionFormat{algor}{%
  \hrulefill\par\offinterlineskip\vskip1pt%
\textbf{#1#2}#3\offinterlineskip\hrulefill}
\DeclareCaptionStyle{algori}{singlelinecheck=off,format=algor,labelsep=space}
\captionsetup[algorithm]{style=algori}

%This version of the thesis template has been updated for the
% February 2017 thesis guidelines provided by the School of Graduate
% and Advanced Studies by David Freund and Daryl DeFord.

%  Some good options are draft and singlespacing, for drafts, and *final*,
%for the final cut, and noheadings, for a printout without headings.
%You can also use the copyright option to add a copyright.  Note:
%final will enforce a bunch of different options, like oneside, 12pt
%and doublespacing, as well as make the margins correct.   Draft, has
%larger margins and is appropriate for twosided printing.

%%%%%%%%%%%%%%%%%%%%%%%%%%%
%%%%    IMPORTANT     %%%%%
%%%%%%%%%%%%%%%%%%%%%%%%%%%
%
%  Because dvips doesn't know/care about the page size of the dvi file
%  its working on, and because its so important that the margins of
%  your thesis are correct you have to make sure that you are using
%  the command dvi -t letter *thesisnamehere*
%
%  If you are using a terminal this is straightforward, but if you are
%  using a tex editing program it make take a little bit of searching
%  before you figure out how to make this work.
%
%  Alternatively, you might consider compiling using pdflatex, which
%  compiles straight to a PDF and doesn't have this problem.

%%%%%%%%%%%%%%%%%%%%%%%%%%%%%%%%%%%%%%%%%%%%%%%%%%
% Additional Packages (add as desired)
%%%%%%%%%%%%%%%%%%%%%%%%%%%%%%%%%%%%%%%%%%%%%%%%%%
\usepackage{amsmath,amssymb,amsthm}

%\usepackage{mkindex} %Uncomment if you would like to have an index.
% Compiling with an index takes more work than compiling without one.
% You will have to look up how to use the index package.

%%%%%%%%%%%%%%%%%%%%%%%%%%%%%%%%%%%%%%%%%%%%%%%%%%
%Formatting
%%%%%%%%%%%%%%%%%%%%%%%%%%%%%%%%%%%%%%%%%%%%%%%%%%
%  Change or add to as desired.
%  These two commands make the first enumerations look like (a)
%  And the second level like (i).
\renewcommand{\labelenumi}{(\alph{enumi})}
\renewcommand{\labelenumii}{(\roman{enumii})}
%  These commands make the section headings Boldface and not all
%  caps. It also removes the chapter numbers
\renewcommand{\chaptermark}[1]{\markboth{{\sc #1}}{{\sc #1}}}
\renewcommand{\sectionmark}[1]{\markright{{\sc \thesection\ #1}}}
%  These commands have lowercase headings with chapter numbers.
%\renewcommand{\chaptermark}[1]{markboth{#1}{}}
%\renewcommand{\sectionmark}[1]{\markright{\thesection\ #1}}

%%%%%%%%%%%%%%%%%%%%%%%%%%%%%%%%%%%%%%%%%%%%%%%%%%
% Theorem Declarations
%%%%%%%%%%%%%%%%%%%%%%%%%%%%%%%%%%%%%%%%%%%%%%%%%%
% A basic set of theorem declarations.  Add or remove as desired.
\newtheorem{prop}{Proposition}[chapter]
\newtheorem{theorem}[prop]{Theorem}
\newtheorem{lemma}[prop]{Lemma}
\newtheorem{corr}[prop]{Corollary}
\theoremstyle{definition}
\newtheorem{definition}[prop]{Definition}
\theoremstyle{remark}
\newtheorem{remark}[prop]{Remark}
\newtheorem{example}[prop]{Example}
\newcommand\dunderline[3][-1pt]{{%
    \sbox0{#3}%
\ooalign{\copy0\cr\rule[\dimexpr#1-#2\relax]{\wd0}{#2}}}}

%%%%%%%%%%%%%%%%%%%%%%%%%%%%%%%%%%%%%%%%%%%%%%%%%%
%Indicies
%%%%%%%%%%%%%%%%%%%%%%%%%%%%%%%%%%%%%%%%%%%%%%%%%%
%This is for an index.  More work is neede for a notation index
%\makeindex

%%%%%%%%%%%%%%%%%%%%%%%%%%%%%%%%%%%%%%%%%%%%%%%%%%
% Macros
%%%%%%%%%%%%%%%%%%%%%%%%%%%%%%%%%%%%%%%%%%%%%%%%%%
% Add your math macros here

%%%%%%%%%%%%%%%%%%%%%%%%%%%%%%%%%%%%%%%%%%%%%%%%%%
% Title Page Information
%%%%%%%%%%%%%%%%%%%%%%%%%%%%%%%%%%%%%%%%%%%%%%%%%%
%  Your personal info goes here!
\title{Action This Day}
\subtitle{The Mathematics and Machinations that Bested the German Enigma}
\author{Jonah Weinbaum}
\date{December 2021}
\degree{Masters of Science}
\field{Computer Science}
\committee{Christophe Hauser}{Sergey Bratus}{Sean Smith}{Asher
Auel}{Peter Winkler}
% \committee[F. Jon Kull, Ph.D.]
\school{Guarini School of Graduate and Advanced Studies}{Dartmouth
College}{Hanover, New Hampshire}
\begin{document}

%%%%%%%%%%%%%%%%%%%%%%%%%%%%%%%%%%%%%%%%%%%%%%%%%%
%Front end of thesis
%%%%%%%%%%%%%%%%%%%%%%%%%%%%%%%%%%%%%%%%%%%%%%%%%%
\frontmatter

\newgeometry{left=1.5in,top=1in,bottom=1in,right=1in}
\maketitle
\restoregeometry

%%%%%%%%%%%%%%%%%%%%%%%%%%%%%%%%%%%%%%%%%%%%%%%%%%
%Abstract
%%%%%%%%%%%%%%%%%%%%%%%%%%%%%%%%%%%%%%%%%%%%%%%%%%

%%%%%%%%%%%%%%%%%%%%%%%%%%%%%%%%%%%%%%%%%%%%%%%%%%
%Table of Contents
%%%%%%%%%%%%%%%%%%%%%%%%%%%%%%%%%%%%%%%%%%%%%%%%%%
\tableofcontents

%Add a list of tables
%\listoftables

%Add a list of figures
%\listoffigures

%%%%%%%%%%%%%%%%%%%%%%%%%%%%%%%%%%%%%%%%%%%%%%%%%%
%Main Portion of Thesis
%%%%%%%%%%%%%%%%%%%%%%%%%%%%%%%%%%%%%%%%%%%%%%%%%%

%%%%%%%%%%%%%%%%%%%%%%%%%
%%  YOUR THESIS HERE!  %%
%%%%%%%%%%%%%%%%%%%%%%%%%
\chapter*{Abstract}
\addcontentsline{toc}{section}{Abstract}

\chapter*{Aims of Thesis}
\addcontentsline{toc}{section}{Aims of Thesis}
It should be made clear that I am not a historian. As such, this
thesis does not aim to be a perfect retelling of historical
narrative; rather, this thesis aims to convey a chronological
progression of cryptographic attacks on Enigma, culminating in the
creation of the Turing-Welchman Bombe, which is the central focus of
this paper. The paper is structured to gradually build the reader's
understanding of the various flaws in the Enigma machine,
illustrating how and why the Bombe came to take the form that it did.
Historical details are included to advance the mathematical narrative
and are, to the best of my ability, confirmed to be historically accurate.
\\\\The first three chapters of this thesis serve as an expository
resource aimed at mathematically inclined readers seeking depth and
rigor regarding cryptographic attacks on Enigma. The final chapter
presents a novel contribution. It critically examines the
mathematical model by which the Bombe's efficacy was described by
Turing and presents an alternative model that better aligns with the
ground-truth operation of the Bombe.

\chapter*{Requisite Background}
\addcontentsline{toc}{section}{Requisite Background}
The subject matter of this paper concerns both the mathematical and
mechanical means by which the Enigma encryption scheme was broken over time.
\\\\With respect to mathematical background, the reader is expected
to have an introductory knowledge of abstract algebra, in particular
a solid grasp of permutation theory. Further, many topics will delve
into probabilistic and combinatoric analysis of language frequency,
keyspaces, and the effectiveness of various methods. For this reason
the reader is expected to have a comfortable understanding of
introductory combinatorics and Bayesian statistics.
\\\\With respect to mechanical background, the reader requires very
limited knowledge of components such as transistors and relays.
Circuit diagrams will be used sparingly or will be abstracted such as
to reduce the need of an electrical engineering background.

\chapter*{Introduction}
\addcontentsline{toc}{section}{Introduction}
While the allied armies fought tooth and nail, in trenches and tanks,
for the fate of the world, another army was amassed in the small
radio factory of Bletchley Park. Mathematicians, engineers,
linguists, and chess players, armed with paper and pencil, wire and
punch-cards, took up their own battle.
\\\\Hitler's best-kept secrets lay hidden behind a math problem. What
follows is the story of some of history's greatest minds and their
work to solve that problem. It is a story of secrecy and spies;
ingenious perspectives, laborious work, and machines built to think
faster than any human ever had.
\\\\At the center of this story is the Enigma machine, a feat of
cryptography so vast in its possible arrangements that it was thought
unbreakable. Breaking it required more than mathematical trickery; it
required coordination, luck, and a cryptographic intelligence team
the likes of which the world had never seen. Winston Churchill saw
this math problem for what it was: the key to saving millions of
lives. Understanding the project's urgency, he issued a memo
demanding that the utmost attention and resources be given to the
Bletchley team. Churchill tagged this memo with a red stamp reserved
for matters of the highest priority. It read:
\begin{center}
  \textbf{Action This Day}
\end{center}
\mainmatter
\chapter{The Enigma}

\begin{chapquote}{Gordon Welchman, \textit{The Hut Six Story Page 52}}
  The Engima, though simple in principle and primitive in many ways,
  presented the cryptanalyst with a dazzling number of possibilities.
\end{chapquote}

The Engima machine was used extensivley by the Germans to encipher
communications prior to and throughout World War II. German
strategems like the \emph{blitzkrieg} required quick radio
communcation, so to ensure that the allied powers did not intercept
signals, they encoded all radio signals using the Enigma machine.
Breaking the Enigma would allow the allied powers to freely intercept
all naval, airforce, and military command -- offering them time to
counter, defend, and retaliate appropriately. Thus, while millions
participated in a war of arms and power, a select group of academics
at Bletchley park engaged in a battle of minds to crack a puzzle
whose solution could save millions of lives.

\section{The Machine}
Throughout this paper, the model \texttt{I} Enigma is chosen to
represent our canonical machine as these were the most common version
used during World War II with over 20000 being produced. Further, it
was used by both the \emph{Heer} (Army), \emph{Luftwaffe} (Air
Force), and the \emph{Kriegsmarine} (Navy) making this a prime target
for attack by cryptanalysts. Many models existed each with varying
layouts, keyspaces, and use-cases; however, the central ideas that
are discussed in this paper can generally be adapted to work on other models.
\\\\At its most basic function, the Enigma (once set up) is a
keyboard, whose letters, when depressed, illuminate bulbs of a
corresponding keyboard layout. The operator presses keys of the
desired plaintext and copies the output of the illuminated bulbs to
get the enciphered text. The actual mechanism of this encipherment
requires several mechanical components in a complex arrangement
\subsubsection{The Plugboard}
\text{}\\
\begin{center}\includegraphics{plugboard.jpg}
\end{center}
Upon a key press, the electrical current corresponding to this letter
is sent to a mechanism known as the plugboard. From an operator's
perspective, the plugboard was a series of ports, one for each
letter, along with 10 cables which could connect these ports. When
two letters are connected via a cable (e.g. A and Z), the plugboard
will send current corresponding to a letter to the opposite letter
(e.g. A goes to Z and vice versa). If no cable is plugged in to a
letter (e.g. D has no cable), then the plugboard simply will return a
current corresponding to this same letter (e.g. D). Assuming all 10
cables are used this means that the plugboard can be represented as a
permutation on 26 letters with a cycle type of $2^{10}1^6$. One such
permutation could be
\begin{center}
  (\texttt{HR})(\texttt{AT})(\texttt{IW})(\texttt{SK})(\texttt{UY})(\texttt{DF})(\texttt{GV})(\texttt{LJ})(\texttt{BQ})(\texttt{MX})(\texttt{C})(\texttt{E})(\texttt{N})(\texttt{O})(\texttt{P})(\texttt{Z})
\end{center}
In general we will denote the permutation corresponding to a plugboard as $P$.

\subsubsection{The Rotors}

\subsubsection{The Reflector}

\section{The Enigma Protocol}
Before describing the internal mechanics of the Enigma, we will first
view the machine as an operator might.
%% https://bletchleypark.org.uk/our-story/enigmas-of-bletchley-park/%%
%% https://www.cryptomuseum.com/crypto/enigma/i/index.htm%%
%% https://www.cryptomuseum.com/crypto/enigma/files/schluessel_m.pdf%^
\\\\Suppose Alice and Bob are two radio operators (between September
1938 and May 1940) who want to communicate securely. Each are
supplied an Enigma machine along with a machine key
(\emph{maschinenschlussel}). THe machine key

Further, each have a copy of the ``\emph{General Regulations for the
Enigma}'' -- a book entailing all the protocols necessary for Alice
and Bob to communicate securely. According to this guide ``all secret
communcations are to be enciphered on the Engima'', in order to do
this, the following guides are necessary external to the machine itself
\begin{itemize}
  \item The general daily key (\emph{Tagesschluessel Allgemein})
  \item The K-Book (\emph{K-Buch})
\end{itemize}

% \\\begin{figure}[h]
%   \begin{center}
%     \resizebox{0.98\textwidth}{!}{
% \begin{tabular}{|c|c|c|c|c|}
% \hline
% \textbf{\emph{\texttt{Datum}}} &
% \textbf{\emph{\texttt{Walzenlage}}} &
% \textbf{\emph{\texttt{Ringstellung}}} &
% \textbf{\emph{\texttt{Steckerverbindungen}}} &
% \textbf{\emph{\texttt{Grundstellung}}} \\
% \hline
% \texttt{31.} & \texttt{IV II I} & \texttt{F T R} & \texttt{HR AT IW
% SK UY DF GV LJ BQ MX}   & \texttt{sfy azy zkq bqi} \\
% \texttt{30.} & \texttt{III V II} & \texttt{Y V P} & \texttt{OR KI
% JV }   & \texttt{iuy swz omo myj} \\
% \texttt{29.} & \texttt{V IV I} & \texttt{O H R} & \texttt{WJ VD PO
% MQ FX ZR NE LG UO BK}   & \texttt{rui kao fqi rwu} \\
%   $\vdots$ & $\vdots$ & $\vdots$ & $\vdots$ & $\vdots$ \\

% \hline
% \end{tabular}}
% \end{center}
%   \caption{Example Engima Key Sheet (September 1938)}
%   \label{fig:enigma_key_sheet}
% \end{figure}
%% https://www.researchgate.net/figure/Enigma-key-book-Photo-from-authentic-German-codebook-From-before-September-1938-as-it_fig2_339932418
% %%

% \\\begin{figure}[h]
%   \begin{center}
%     \resizebox{0.98\textwidth}{!}{
% \begin{tabular}{|c|c|c|c|c|}
% \hline
% \textbf{\emph{\texttt{Datum}}} &
% \textbf{\emph{\texttt{Walzenlage}}} &
% \textbf{\emph{\texttt{Ringstellung}}} &
% \textbf{\emph{\texttt{Steckerverbindungen}}} &
% \textbf{\emph{\texttt{Kenngruppen}}} \\
% \hline
% \texttt{31.} & \texttt{V II IV} & \texttt{17 09 02} & \texttt{KT AJ
% IV UR NY HZ GD XF PB CQ}   & \texttt{sfy azy zkq bqi} \\
% \texttt{30.} & \texttt{I III V} & \texttt{22 12 10} & \texttt{UE PL
% AY TB ZH WM OJ DC KN SI}   & \texttt{iuy swz omo myj} \\
% \texttt{29.} & \texttt{V IV II} & \texttt{04 01 25} & \texttt{WJ VD
% PO MQ FX ZR NE LG UO BK}   & \texttt{rui kao fqi rwu} \\
% \texttt{28.} & \texttt{II III IV}  & \texttt{05 03 12} & \texttt{HR
% TJ LD IO CN GX QK PZ WS AF}   & \texttt{ioy kjv yko fpz} \\
% $\vdots$ & $\vdots$ & $\vdots$ & $\vdots$ & $\vdots$ \\
% \hline
% \end{tabular}}
% \end{center}
%   \caption{Mock Enigma Key Sheet for April 1943.}
%   \label{fig:enigma_key_sheet}
% \end{figure}

\chapter{The Polish Bomba}
%% https://www.cryptocellar.org/pubs/ukwa.pdf%%
%% https://www.cryptocellar.org/enigma/files/rejewski-paper.pdf %% 

Rejewski was saddled with arguably the most complex discoveries in Enigma code breaking. Not only did he need to determine a means to recover daily keys from limited intellegence supplied by ?? but he additionally needed to recover the wirings of the rotors themselves.
\section{Characteristics}
Rejewksi quickly determined the protocol by which messages were enciphered -- in fact, he stated the this protocol was ``obvious.'' We will see that purely with knowledge of this procudure and some military intellegence, Rejewski was able to determine the rotor wirings necessary to make further cryptanalysis possible.
\\\\Consider the first six letters transmitted according to our encryption producedure. Operator Alice has some three letter private key (say \texttt{XYZ}) which she encodes twice with the machine settings specified by her key sheet. This will give us six encrypted letters $\sigma_1(\texttt{X})\sigma_2(\texttt{Y})\sigma_3(\texttt{Z})\sigma_4(\texttt{X})\sigma_5(\texttt{Y})\sigma_6(\texttt{Z})$. Suppose these six letters are given as
\begin{center}
	\texttt{ABC} \texttt{DEF}
\end{center}
That is
\begin{align*}
	\sigma_1(\texttt{X}) & = \texttt{A} \\
	\sigma_2(\texttt{Y}) & = \texttt{B} \\
	\sigma_3(\texttt{Z}) & = \texttt{C} \\
	\sigma_4(\texttt{X}) & = \texttt{D} \\
	\sigma_5(\texttt{Y}) & = \texttt{E} \\
	\sigma_6(\texttt{Z}) & = \texttt{F} \\
\end{align*}
Since each $\sigma_i$ is represented by 13 disjoint transpositions we can deduce, for example, that
\[
	\sigma_1\sigma_4(D) = \sigma_1(X) = A.
\]
With a sufficient set of hexagrams from gathered messages, we could then fully deduce the permutation $\sigma_1\sigma_4$. Further, we could recover $\sigma_2\sigma_5$ and $\sigma_3\sigma_6$. Rejewski refered to these paired permutations as {\bf{characteristics}}. In practice, such recovered characteristics may look like
\begin{align*}
	\sigma_1\sigma_4 & = (\texttt{DVPFKXGZYO})(\texttt{EIJMUNQLHT})(\texttt{BC})(\texttt{RW})(\texttt{A})(\texttt{S}) \\
	\sigma_2\sigma_5 & = (\texttt{BLFQUEOUM})(\texttt{HJPSWIZRN})(\texttt{AXT})(\texttt{CGY})(\texttt{D})(\texttt{K}) \\
	\sigma_3\sigma_6 & = (\texttt{ABVIKTJGFCQNY})(\texttt{DUZREHLXWPSMO})
\end{align*}
Rejewski noted a key structural similarity between all such characteristics recovered in this fashion: in each characteristic, cycles of the same length appear in pairs.
\\\\To see why this happens consider the following lemma,
\begin{lemma}
	\label{cillies}
	Suppose $(\alpha\beta)$ appears in $\sigma_i$ for $i\in\{1,2,3\}$. Then $\alpha$ and $\beta$ will appear in disjoint cycles of $\sigma_i\sigma_{i+3}$ of equal length.
\end{lemma}
\begin{proof}
	%% chrome-extension://efaidnbmnnnibpcajpcglclefindmkaj/https://www.math.ias.edu/files/wam/rejewski.pdf %%
	We begin by noting that if $(\alpha\beta)$ is in $\sigma_{i+3}$ then $\pi$ contains fixed points at $\alpha$ and $\beta$ and our claim is true. Then without loss of generality we can arrange $\sigma_{i}$ and $\sigma_{i+3}$ (non-exhaustively) in the following way:
	\[
		\setlength{\arraycolsep}{15pt}
		\begin{array}{cc}
			\sigma_i      & \sigma_{i+3} \\
			\hline
			(\alpha\beta) & (\beta x_1)  \\
			(x_1 x_2)     & (x_2 x_3)    \\
			\vdots        & \vdots       \\
			(x_{k-1} x_k) & (x_k \alpha)
		\end{array}
	\]
	Then the product $\sigma_i\sigma_{i+3}$ will be
	\[
		\sigma_1\sigma_{i+3} = (\alpha x_1 x_3 \dots x_{k-1} )(x_k x_{k-2} \dots x_2 \beta)
	\]
	and thus $\alpha$ and $\beta$ end up in disjoint cycles of equal length.
\end{proof}

This lemma has several consequences.
\begin{itemize}
	\item A characteristic like $\sigma_1\sigma_4$ which has two singletons $\texttt{A}$ and $\texttt{S}$ (in this context they are refered to as {\bf{females}}) then both $\sigma_1$ and $\sigma_4$ must have the transposition $(\texttt{AS})$.
	\item A characteristic like $\sigma_3\sigma_6$ (two disjoint 13 cycles) reduces the space of possible $\sigma_3$'s to just 13 permutations which will take the form
	      \begin{center}
		      $(\texttt{AD})(\texttt{BO})(\texttt{VM})\dots(\texttt{YU})$ \\
		      $(\texttt{AU})(\texttt{BD})(\texttt{VO})\dots(\texttt{YZ})$ \\
		      $\vdots$                                       \\
		      $(\texttt{AO})(\texttt{BM})(\texttt{VS})\dots(\texttt{YD})$
	      \end{center}
	      and similarly for $\sigma_6$.
\end{itemize}
Thus with absolutely no knowledge of the rotor wirings or the daily key, we can already tractibly compute a searchable space of $\sigma_i$'s. To determine which $\sigma_i$ is the correct one, we make use of the most prevalent bug in cryptography -- operator error.

\subsection{Cillies}
Enigma operators were instructed to construct random trigrams for their message keys -- likely to prevent frequency analysis attacks on the hexagrams beginning messages; However, operators often chose the same trigrams for each message. Some examples might inclide
\begin{itemize}
	\item Initials or first letters of the operator's spouse. For example, \texttt{CIL} perhaps deriving from the name ``Cecelia'' being shortened to ``cillie''. Allegadely for this reason, poor selections of trigrams from operators became known as {\bf{cillies}}.
	\item The same letter encoded three times such as \texttt{JJJ}
	\item Locations such as \texttt{LON} representing London.
	\item Later, Bletchley Park cryptanalyst John Herivel noticed that many operators selected message keys that were very close to the provided ring settings potentially allowing for a quick means of determining the ring settings.

	      %% Battle of wits the complete story of codebreaking in World War II PAGE 143 %% 
\end{itemize}
By keeping track of various radio stations which used cillies. Rejewski had a reasonable guess as to what the message key used for a particular ciphertext were. Suppose we recieved three hexagrams originating from a radio station which often used \texttt{JJJ} as their message key:
\begin{center}
	\texttt{SUG SMF}\\
	\texttt{SJM SPO}\\
	\texttt{SYX SCW}.
\end{center}

We can then compare these message keys against our possibilites for $\sigma_i$s to determine if $\texttt{JJJ}$ could have in fact been used to encipher these hexagrams. For example, the first hexagram could not have been used since \texttt{J} and \texttt{G} lie in the same cycle of $\sigma_3\sigma_6$ which would contradict lemma \ref{cillies}. Contiuing in this fashion we may hypothesize that the third hexagram was likely an enciphering of the repeated letter $\texttt{J}$.
\\\\To confirm this hypothesis we can use the supposition to further reduce the possible $\sigma_i$s. In fact, in our case, such a hypothesis would completely determine which $\sigma_3$ and $\sigma_6$ were being used and for the remaining $\sigma_i$s we are only left with a small set of options. We can then confirm our hypothesis by trying such $\sigma_i$s on hexagrams from other radio stations suspected of using cillies. If we find that $\texttt{PPP}$ is the message key corresponding to our deduced $\sigma_i$s we have reason to believe that our initial hypothesis was correct. If our hexagram corresponded to a key \texttt{PPA} we may only need to select a new choice of possible $\sigma_i$s to instead produce the expected cilly. In this way, we can recover each $\sigma_i$ and thus recover any transmission's message key -- all without any knowledge of internal wirings, plugboard settings, or daily keys.
\section{Wiring Recovery}

Equipped with a means to determine each $\sigma_i$, Rejewski set himself to finding the internal wirings of the rotors. The full recovery of rotor wirings and turovers is out of scope of this paper; However, we will provide a brief description to illustrate that with minimal intellegence information, entire rotor wirings were able to be deduced.
\\\\We begin by expanding $\sigma_i$ to
\[
	\sigma_i = S^{-1}P^{-(x+i)}N^{-1}P^{x+i}M^{-1}L^{-1}RLMP^{-(x+i)}NP^{x+i}S
\]
where $x$ accounts for the initial starting position of the rightmost rotor.
Since $M^{-1}L^{-1}RLM$ does not change between $\sigma_i$s we will denote this permutation $Q$ thus simplifying our earlier expression to
\[
	\sigma_i = S^{-1}P^{-(x+i)}N^{-1}P^{x+i}QP^{-(x+i)}NP^{x+i}S
\]
Rejewski knew the plugboard settings for two whole months, so he began shifting knowns and unknowns to opposite sides. Shifting things around gives us
\begin{align*}
	                    & \sigma_i = S^{-1}P^{-(x+i)}N^{-1}P^{x+i}QP^{-(x+i)}NP^{x+i}S     \\
	\Rightarrow\text{ } & S\sigma_i S^{-1} = P^{-(x+i)}N^{-1}P^{x+i}QP^{-(x+i)}NP^{x+i}    \\
	\Rightarrow\text{ } & P^{(x+i)}S\sigma_i S^{-1}P^{-(x+i)} =  N^{-1}P^{x+i}QP^{-(x+i)}N
\end{align*}
To further simplify notation we will then define ${\rho_i} \coloneq P^{(x+i)}S\sigma_i S^{-1}P^{-(x+i)}$. Then we have now have
\[
	\rho_i = N^{-1}P^{x+i}QP^{-(x+i)}N
\]
where $\rho_i$s are known from the key sheets Rejewski had access to. We will now eliminate this equations dependence on $Q$ by considering pairs of $\rho_i$ and $\rho_{i+1}$
\begin{align*}
	\rho_i\rho_{i+1} & = N^{-1}P^{x+i}QP^{-(x+i)}NN^{-1}P^{x+i+1}QP^{-(x+i+1)}N \\
	                 & = N^{-1}P^{x+i}QP^{-(x+i)}P^{x+i+1}QP^{-(x+i+1)}N        \\
	                 & = N^{-1}P^{x+i}QPQP^{-(x+i+1)}N                          \\
	                 & =N^{-1}P^{x+i}(QPQP^{-1})P^{-{x+i}}N
\end{align*}
Each $\rho_i\rho_{i+1}$ shares the common subexpression $QPQP^{-1}$. We can eliminate this subexpression by noting
\begin{align*}
	\rho_{i+1}\rho_{i+2} & = N^{-1}P^{x+i+1}(QPQP^{-1})P^{-{x+i+1}}N                                                     \\
	                     & = N^{-1}P^{x+i+1}(P^{-(x+i)}NN^{-1}P^{x+i})(QPQP^{-1})(P^{-(x+i)}NN^{-1}P^{x+i})P^{-{x+i+1}}N \\
	                     & = N^{-1}P^{x+i+1}P^{-(x+i)}N(N^{-1}P^{x+i}QPQP^{-1}P^{-(x+i)}N)N^{-1}P^{x+i}P^{-{x+i+1}}N     \\
	                     & = N^{-1}P^{x+i+1}P^{-(x+i)}N(\rho_i\rho_{i+1})N^{-1}P^{x+i}P^{-{x+i+1}}N                      \\
	                     & = N^{-1}P^{-1}N(\rho_i\rho_{i+1})N^{-1}PN                                                     \\
\end{align*}
We now have a relationship between each $\rho_i\rho_{i+1}$ and $\rho_{i+1}\rho_{i+2}$ by conjugating by $N^{-1}PN$. Now recall from theorem \ref{IDK} that if $\rho_i\rho_{i+1}$ has a reasonbaly large cycle structure, then there are only a limited number of possible permutations for $N^{-1}PN$. The relationship between $\rho_{i+1}\rho_{i+2}$ and $\rho_{i+2}\rho_{i+3}$ will further reduce these possibilities since of course $N^{-1}PN$ must be the same between these two relationships. Eventually, we can fully deduce $N^{-1}PN$. Finally, this will give us $26$ possibilites for $N$ representing its $26$ intial starting positions, and thus we can recover $N$ itself representing the internal wiring of the rigthmost rotor.
\\\\In a similar fashion we can recover the remaining rotor wirings though often with the help of other mathematical tricks, not to mention military intellegence and luck. From this point forward, we will assume that the cryptanalyst now has access to the wirings of all the rotors as well as the reflector.

\section{The Grill Method}

We now have deduced all of the rotors $N$, $M$, $L$, $R$ and each $\sigma_i$. We will use this information to recover the daily keys. For the moment, let us assume $S$ is the identity permutation. Then rearranging $\sigma_i$ we get
\begin{align}
	Q = P^{-(x+i)}NP^{x+i}\sigma_iP^{-(x+i)}N^{-1}P^{x+i} \label{eq:q_eq}
\end{align}
Since $Q$ must be the same for each such equation involving $\sigma_i$ we will devise a manual way to deduce $Q$. We can precompute $N$, $P^{-1}NP$, $\dots$, $P^{4}NP^{-4}$ and arrange them in a large sheet called the {\bf{bottom sheet}}.
Then for each $\sigma_i$ we can write out $\sigma_i$ with a slit beneath it to allow space for each possible $P^{-k}NP^k$. We will illustrate this for $\sigma_1$ and denote the spaces beneath with periods.
\begin{align*}
	\texttt{|}          & \texttt{ABCDEFGHIJKLMNOPQRSTUVWXYZ} \texttt{|} \\
	\sigma_1\texttt{ |} & \texttt{SRWIVHNFDOLKYGJTXBAPZECQMU} \texttt{|} \\
	\texttt{|}          & \texttt{..........................} \texttt{|}
\end{align*}
We will call this the {\bf{top sheet}}. By sliding the bottom sheet beneath the top sheet we can test various values of $P^{-k}NP^{k}$ to see what values of $Q$ they give. For example, for $k = 0$ we have
\begin{align*}
	\texttt{|}          & \texttt{ABCDEFGHIJKLMNOPQRSTUVWXYZ} \texttt{|} \\
	\sigma_1\texttt{ |} & \texttt{SRWIVHNFDOLKYGJTXBAPZECQMU} \texttt{|} \\
	N \texttt{ |}       & \texttt{KJPZYDTIOHXCSGUBRNWFMVEQLA} \texttt{|}
\end{align*}
We know that $Q(\texttt{A}) = (P^{-k}NP)\sigma_i(P^{-k}NP)^{-1}(\texttt{A})$. Therefore to compute $Q(\texttt{A})$ we first begin at $\texttt{A}$ in our bottom row to find that $N^{-1}(\texttt{A})=\texttt{Z}$. We now map $\texttt{Z}$ through $\sigma_i$ by finding $\texttt{Z}$ on the top row and seeing where it lands in the middle row, thus giving $\sigma_iN^{-1}(\texttt{A}) = \texttt{U}$. Finally to see where $N$ maps $\texttt{U}$ by finding $\texttt{U}$ on the top row and seeing where it lands in the middle row, thus giving $Q(\texttt{A}) = N\sigma_iN^{-1}(\texttt{A}) = M$. Continuing in this fashion we can get a candidate $Q$ generated by the guess that $\sigma_i$ aligned with $N$ in the sheet. Recall that $Q$ must be consistent between each equation \ref{eq:q_eq}. To check this we can simply construct our top sheet so that each $\sigma_i$ is placed over one another each with a slot beneath it. We can now get candidate $Q$s for each $\sigma_i$. If we find that our offset of the bottom sheet creates inconsistencies we can move the bottom sheet up by one until we find candidates that are consistent between each $\sigma_i$.
\\\\If $S$ were truly the identity then we would ultimately find an offset that generates identical $Q$s for each $\sigma_i$. Of course, $S$ will not be the identity. Thus, instead of looking for perfect consistency between each $Q$ we are only looking for relative consistency between each $Q$ where perhaps a majority send one letter to another. By comparing $Q$s generated by each $\sigma_i$ we can deduce with reasonable certainty the value of the true $Q$ by just considering where the majority of the $Q$s map \texttt{A}, \texttt{B}, and so on. Further, the offset in the bottom sheet which generated the most consistent permutations $Q$ will give us the absolute position of the rightmost rotor. Additionally, using this method, we can see where the $Q$s fail to line up to determine which letters are steckered and which are unsteckered, along with some of the steckerings themselves. Such a method was called the {\bf{grill method}} and required tedious work and had many possibilites for mistakes.
\\\\To determine the absolute positions of the remaining two rotors we can simply enumerate all $26^2 = 676$ positions of the left two rotors (for both possible orderings of the rotors) until $Q$ is produced. In practice, a catalogue was eventually compiled which associated each $Q$ to a corresponding position and ordering of the left two rotors. Now equipped with each absolute rotor position, we still must determine the ring settings.
\\\\To determine the ring settings we make use of another operator error. Often messages began with the letters \texttt{ANX} which is the German word ``to'' along with \texttt{X} denoting a space. We could therefore set the ring setting to its default lcoation and then brute force all $26^3$ possible rotor positions until the first letters in the deciphered message were \texttt{ANX}. Once we knew the absolute position at which this occured, along with the message key, we can immediately determine what the ring settings must have been to produce this message -- since the ring setting and rotor position have inverse effects on the rotor permutation.
\\\\One key point regarding the grill method is that at the time when it was in use only $6$ jacks were used in the plugboard. This means that the $Q$s generated by each $\sigma_i$ were relatively similar since the permutation $S$ had less impact on average. However, the Germans eventually used up to $10$ jacks making the grill method infeasible since we can no longer determine confidently the true value of $Q$. This fact will later necessitate the development of an additional tool called the {\bf{Bomba}} later discussed in this chapter.

\section{The Clock Method}

At this point Rejewski and his team were able to recover daily keys, but the above methods are extremely slow and inefficient. Many optimizations were made over the years. We will now examine one particular optimization.
\\\\While we do know the wirings of the rotors, we do not know when we begin our cryptanalysis what the order of these rotors were. At the time there were only three rotors in use (\texttt{I}, \texttt{II}, \texttt{III}) so one could simply try all 3 rotors as the rightmost one and repeat the above analysis. Of course, this makes the above method 3 times slower. In practice, early Enigma daily keys kept rotor positions the same for an entire 3 month period meaning this analysis did not need to be done too frequently.
\\\\However, Jerzy Różycki worked out an efficient method to determine the rightmost rotor which he called the {\bf{clock method}}. The clock method attempted to determine where the turnover notch was for the rightmost rotor. From this we could immediately determine which of the three rotors was used.

\subsection{Index of Coincidence}
In a string of random text from a 26 letter alphabet we get a uniform distribution of letters. However, text which encodes a language does not generate a uniform distribution. Distributions for various languages have been well studied and are the information needed for frequency analysis. Having a non-uniform distribution also implies that when we align two pieces  of text which encodes a language (call them $\texttt{T}_A$ and $\texttt{T}_B$), the chance that a letter from $\texttt{T}_A$ will align at the same position with a letter from $\texttt{T}_B$ is non-uniform. We can therefore detect if  $\texttt{T}_A$ and $\texttt{T}_B$ are encoded with the same polyalphabetic cipher by counting the number of aligned letters between them (called {\bf{coincidences}}) and seeing if they represent a non-uniform frequency.
\\\\To make use of this property, we can select two messages with message keys whose first two letters coincided (e.g. \texttt{XYA} and \texttt{XYF}). This meant that both messages were encoded with the only difference being in their rightmost rotor. If at some point during encoding the first message, its rotors align with the rotors used to encode the second message, we would expect to see the number of coincidences between the messages to suddenly spike significantly, in this way, we can detect when the two messages had their rotors align. This does not work, however, if turnover occurs then the two messages will not align in their rotor positions and we will see a random distribution of coincidences. If we align the message generated by $\texttt{XYA}$ on top of the message generated by $\texttt{XYF}$ we can slide the bottom message until is 5 positions further than the top message,
\begin{align*}
	 & \texttt{XYA}: \texttt{PASLK XASSP AUSDK XPVNW UULVT LWKRE KGUQO DSUKV ZOLMZ ZHYBF}                                                                                                                                                      \\
	 & \texttt{XYF}: \texttt{ }\texttt{ }\texttt{ }\texttt{ }\texttt{ }\texttt{ }\texttt{BAWXV ETTOP JZHXL VWGGQ MWDII OEHQO YSLRB IYLGB CHBYT }                                                                                               \\
	 & \ \ \ \ \ \ \ \ \ \ \ \ \ \ \ \ \ \ \texttt{*}\ \ \ \ \ \ \ \ \ \ \ \ \ \ \ \ \ \ \ \ \ \ \ \ \ \ \ \ \ \ \ \ \ \ \ \ \ \texttt{*}\ \ \ \ \ \ \ \ \ \ \ \texttt{**}\ \ \ \texttt{*}\ \ \ \ \ \ \ \ \ \texttt{*}\ \ \ \ \ \ \ \texttt{*}
\end{align*}
at this point we would expect the texts to both be encoded by $\texttt{XYF}$ and the number of coincidences (indicated by \texttt{*}) should spike as we see in the above diagram. If this does not occur, this means no turnover occured between $\texttt{A}$ and $\texttt{F}$. This, for example, eliminates rotor \texttt{II} as a candidate since its turnonver occurs when \texttt{E} is displayed in the window. We can gain further information by performing the above procecdure but now with the message encoded by $\texttt{XYF}$ on the top and the message encoded by $\texttt{XYA}$ on the bottom. In this way we can determine which rotor was being used. The clock method was a precursor to a method we will discuss later known as Banburismus. The important takeaway is that language frequency analysis may not be strong enough to decode Enigma messages, it may be strong enough to determine elements of the key like rotor choice or ordering.

\section{The Cyclometer}

These manual methods of decryption became increasingly difficult as German operators were instructed to use more plugboard jacks and increased the rate at which daily keys changed. Rejewski wanted to produce a mechanical means of performing a similar deduction. He returned to the characteristics associated with a particular day's key. He noticed that the cycle structure of a characteristic did not regularly repeat (if at all). Then we perhaps could create a ``fingerprint'' of a key by noting the cycle structure of all three characteristics associated to the key. Now that Rejewski had the internal rotor wirings, he could build a machine that could immediately produce the cycle structure of characteristics for a given setting.
\\\\Let us first understand the manual implementation of such a deduction. Assuming an identity plugboard $S$ we can use our internal rotor wirings to know the exact permutaition $\sigma_1$ and $\sigma_3$ for any initial rotor position. In specific, suppose we have

\begin{center}
	\[
		\left(
		\begin{array}{llllllllllllllllllllllllll}
				\texttt{A} & \texttt{B} & \texttt{C} & \texttt{D} &
				\texttt{E} & \texttt{F} & \texttt{G} & \texttt{H} &
				\texttt{I} & \texttt{J} & \texttt{K} & \texttt{L} &
				\texttt{M} & \texttt{N} & \texttt{O} & \texttt{P} &
				\texttt{Q} & \texttt{R} & \texttt{S} & \texttt{T} &
				\texttt{U} & \texttt{V} & \texttt{W} & \texttt{X} &
				\texttt{Y} & \texttt{Z}                             \\
				\texttt{P} & \texttt{T} & \texttt{K} & \texttt{X} &
				\texttt{R} & \texttt{Z} & \texttt{Q} & \texttt{S} &
				\texttt{W} & \texttt{M} & \texttt{C} & \texttt{O} &
				\texttt{J} & \texttt{Y} & \texttt{L} & \texttt{A} &
				\texttt{G} & \texttt{E} & \texttt{H} & \texttt{B} &
				\texttt{V} & \texttt{U} & \texttt{I} & \texttt{D} &
				\texttt{N} & \texttt{F}
			\end{array}
		\right)
	\]
	$\sigma_1$
\end{center}
\begin{center}
	\[
		\left(
		\begin{array}{llllllllllllllllllllllllll}
				\texttt{A} & \texttt{B} & \texttt{C} & \texttt{D} &
				\texttt{E} & \texttt{F} & \texttt{G} & \texttt{H} &
				\texttt{I} & \texttt{J} & \texttt{K} & \texttt{L} &
				\texttt{M} & \texttt{N} & \texttt{O} & \texttt{P} &
				\texttt{Q} & \texttt{R} & \texttt{S} & \texttt{T} &
				\texttt{U} & \texttt{V} & \texttt{W} & \texttt{X} &
				\texttt{Y} & \texttt{Z}                             \\
				\texttt{J} & \texttt{W} & \texttt{V} & \texttt{R} &
				\texttt{O} & \texttt{S} & \texttt{U} & \texttt{Y} &
				\texttt{Z} & \texttt{A} & \texttt{T} & \texttt{Q} &
				\texttt{X} & \texttt{P} & \texttt{E} & \texttt{N} &
				\texttt{L} & \texttt{D} & \texttt{F} & \texttt{K} &
				\texttt{G} & \texttt{C} & \texttt{B} & \texttt{M} &
				\texttt{H} & \texttt{I}
			\end{array}
		\right)
	\]
	$\sigma_3$
\end{center}
Then to find the permutation $\sigma_1\sigma_3$ we might do the following. We begin with \texttt{A}. We first run \texttt{A} through $\sigma_3$ to get \texttt{J}. We then run \texttt{J} through $\sigma_1$ to get \texttt{M}. Thus we see $\sigma_1\sigma_3(\texttt{A}) = \texttt{M}$. Now we would continue with \texttt{M} sending it back into $\sigma_3$. Eventually after jumping back and forth between $\sigma_1$ and $\sigma_3$ we we will have encountered all letters contained in the same cycle as \texttt{A}. We do, however, encounter some other letters as well. This method is effectively the same method used in lemma \ref{cillies} to find the two disjoint cycles of equal length produced in a characteristic. In fact, by switching back and forth between $\sigma_1$ and $\sigma_3$ we will find alternating elements of the pair of cycles of equal length which contain \texttt{A}. We can mechanize this process via the following circuit:
\begin{center}
	\begin{tikzpicture}[thick, scale=0.6, every node/.style={scale=0.7}]
		% Draw the box
		\draw[fill=lightgray] (2-5,-1.5-1) rectangle (4-5,2.5-1) node[midway] {};

		\node at (3-5, -2-1) {$Q_{y,z}$};

		\draw[-] (2-3, 2-1) |- ++(-0.75, 0) |- (2-3, 0-1);
		\draw[-] (2-3, 1-1) |- ++(-1, 0) |- (2-3, -1-1);

		\draw[fill=lightgray] (2-3,-1.5-1) rectangle (4-3,2.5-1) node[midway] {};

		\node at (3-3, -2-1) {$N_x$};

		% Draw the lines inside the box to represent the mapping
		\draw[-] (2-3, 2-1) -- (4-3, 0-1);
		\draw[-] (2-3, 1-1) -- (4-3, -1-1);
		\draw[-] (2-3, 0-1) -- (4-3, 2-1);
		\draw[-] (2-3,-1-1) -- (4-3, 1-1);

		% Draw the lines inside the box to represent the mapping
		\draw[-] (2-1, -1-1) -- (4+3, -1-1);
		\draw[-] (2-1, 0-1) -- (4+3, 0-1);
		\draw[-] (2-1, 1-1) -- (4+3, 1-1);
		\draw[-] (2-1,2-1) -- (4+3, 2-1);

		\draw[fill=lightgray] (2+3,-1.5-1) rectangle (4+3,2.5-1) node[midway] {};

		\node at (3+3, -2-1) {$N_{x+3}$};

		% Draw the lines inside the box to represent the mapping
		\draw[-] (2+3, 1-1) -- (4+3, 2-1);
		\draw[-] (2+3, 2-1) -- (4+3, 1-1);
		\draw[-] (2+3, 0-1) -- (4+3, -1-1);
		\draw[-] (2+3,-1-1) -- (4+3, 0-1);

		\draw[fill=lightgray] (2+5,-1.5-1) rectangle (4+5,2.5-1) node[midway] {};

		\node at (3+5, -2-1) {$Q_{y,z}$};

		\draw[-] (2+5, 2-1) |- ++(0.75, 0) |- (2+5, 0-1);
		\draw[-] (2+5, 1-1) |- ++(1, 0) |- (2+5, -1-1);

		\node at (2+1, 2.3-1) {$\texttt{A}$};
		\node at (2+1, 1.3-1) {$\texttt{B}$};
		\node at (2+1, 0.3-1) {$\texttt{C}$};
		\node at (2+1, -0.6-1) {$\texttt{D}$};

	\end{tikzpicture}
\end{center}

In this circuit $N_x$ represents the rightmost rotor at some position $x$ and $Q_{y,z}$ represents our fictious rotor (consisting of $M$, $L$, and $R$) at a fixed position $y$ and $z$. Note that right hand rotor in the permutation depicted on the right is 3 steps further than the right hand rotor depicted on the left. When we now apply a current at \texttt{A} the current will travel back and forth through $N_xQ_{y,z}$ and $N_{x+3}Q_{y,z}$ and reach all the letters contained in the pair of cycles of equal length which contain \texttt{A}, that is, the number of wires electrified is double the length of the cycle containing \texttt{A}. Thus, we can quickly determine the length of the cycles in a permutation $\sigma_1\sigma_{3}$ at any given position of rotors ${x}$, $y$, and $z$. We can now rotate through all $26^3$ positions of the rotors and create a catalogue of all characteristics $\sigma_1\sigma_3$ (note that $\sigma_2\sigma_4$ and $\sigma_5\sigma_6$ can be retrieved by just looking at the next two catalogue entries). We must also consider all 6 orderings of rotors \texttt{I}, \texttt{II}, and \texttt{III}. Thus we can laboriously generate a catalogue of all rotor orders and settings consisting of $105456$ entries.
\\\\Now equipped with this catalogue, the cryptanalyst could use the traffic from a given day to determine the characteristics, look up the cycle types in the catalogue, and determine the rotor order and absolute rotor positions. The plugboard settings could then be determined by comparing the characteristics recovered from the traffic against the characteristics generated by the discovered rotor positions with no plugboard jacks inserted. Finally the ring position could be recovered as discussed before by using known plaintext from the message. By this point messages stopped beginning with \texttt{ANX} and now began with known tetragrams from a codebook, thus making such an attack still feasible. This ultimately reduced the time to find a day's settings to roughly $15$ minutes.
\\\\Note that the machine made finding cycles in permutations instantaneous by connecting each cycle in its own disjoint electrical circuit. This first attempt at mechanizing the process of decryption is a very early predacessor to the primary topic of this paper, the Bombe. This method of enumerating cycles, as we will see, is the core of the Bombe's primary function.
\section{The Bomba}

\begin{circuitikz}
	% Draw the box
	\draw[fill=lightgray] (2-5,-1.5-1) rectangle (4-5,2.5-1) node[midway] {};

	\node at (3-5, -2-1) {$Q_{y,z}$};

	\draw[-] (2-3, 2-1) |- ++(-0.75, 0) |- (2-3, 0-1);
	\draw[-] (2-3, 1-1) |- ++(-1, 0) |- (2-3, -1-1);

	\draw[fill=lightgray] (2-3,-1.5-1) rectangle (4-3,2.5-1) node[midway] {};

	\node at (3-3, -2-1) {$N_x$};

	% Draw the lines inside the box to represent the mapping
	\draw[-] (2-3, 2-1) -- (4-3, 0-1);
	\draw[-] (2-3, 1-1) -- (4-3, -1-1);
	\draw[-] (2-3, 0-1) -- (4-3, 2-1);
	\draw[-] (2-3,-1-1) -- (4-3, 1-1);


	% Wiring exiting 
	\draw (1,2-1) to (1.75,2-1);
	\draw[line width=1.25pt] (1.75, 2-1) to ++(60:0.5);
	\fill (1.75, 2-1) circle (2pt); % dot
	\fill (1.75,2-1) ++(60:0.5) circle (2pt); % dot
	\draw (2.25,2-1) circle (2pt); % dot
	\draw (2.3,2-1) to (3,2-1);
	\draw (1,1-1) to (1.75,1-1);
	\draw[line width=1.25pt] (1.75,1-1) to (2.25,1-1);
	\fill (1.75, 1-1) circle (2pt); % dot
	\fill (2.25, 1-1) circle (2pt); % dot
	\draw (2.25,1-1) to (4.5,1-1);
	\draw (1,0-1) to (1.75,0-1);
	\draw[line width=1.25pt] (1.75,0-1) to (2.25,0-1);
	\fill (1.75, 0-1) circle (2pt); % dot
	\fill (2.25, 0-1) circle (2pt); % dot
	\draw (2.25,0-1) to (6,0-1);
	\draw (1,-1-1) to (1.75,-1-1);
	\draw[line width=1.25pt] (1.75,-1-1) to (2.25,-1-1);
	\fill (1.75, -1-1) circle (2pt); % dot
	\fill (2.25, -1-1) circle (2pt); % dot
	\draw (2.25,-1-1) to (7.5,-1-1);

	% Draw the box
	\draw[fill=lightgray] (2-5,-1.5-7) rectangle (4-5,2.5-7) node[midway] {};

	\node at (3-5, -2-7) {$Q_{y,z}$};

	\draw[-] (2-3, 2-7) |- ++(-0.75, 0) |- (2-3, 0-7);
	\draw[-] (2-3, 1-7) |- ++(-1, 0) |- (2-3, -1-7);

	\draw[fill=lightgray] (2-3,-1.5-7) rectangle (4-3,2.5-7) node[midway] {};

	\node at (3-3, -2-7) {$N_{x+3}$};

	% Draw the lines inside the box to represent the mapping
	\draw[-] (2-3, 2-7) -- (4-3, 1-7);
	\draw[-] (2-3, 1-7) -- (4-3, 2-7);
	\draw[-] (2-3, 0-7) -- (4-3, -1-7);
	\draw[-] (2-3,-1-7) -- (4-3, 0-7);

	% Wiring exiting 
	\draw (1,2-7) to (1.75,2-7);
	\draw[line width=1.25pt] (1.75, 2-7) to ++(60:0.5);
	\fill (1.75, 2-7) circle (2pt); % dot
	\fill (1.75,2-7) ++(60:0.5) circle (2pt); % dot
	\draw (2.25,2-7) circle (2pt); % dot
	\draw (2.3,2-7) to (3,2-7);
	\draw (1,1-7) to (1.75,1-7);
	\draw[line width=1.25pt] (1.75,1-7) to (2.25,1-7);
	\fill (1.75, 1-7) circle (2pt); % dot
	\fill (2.25, 1-7) circle (2pt); % dot
	\draw (2.25,1-7) to (4.5,1-7);
	\draw (1,0-7) to (1.75,0-7);
	\draw[line width=1.25pt] (1.75,0-7) to (2.25,0-7);
	\fill (1.75, 0-7) circle (2pt); % dot
	\fill (2.25, 0-7) circle (2pt); % dot
	\draw (2.25,0-7) to (6,0-7);
	\draw (1,-1-7) to (1.75,-1-7);
	\draw[line width=1.25pt] (1.75,-1-7) to (2.25,-1-7);
	\fill (1.75, -1-7) circle (2pt); % dot
	\fill (2.25, -1-7) circle (2pt); % dot
	\draw (2.25,-1-7) to (7.5,-1-7);


	% Circuit wires for A
	\draw (3.5, -11) to (3.5,-1.5-7);
	\draw (3.5,-1.5-7) to[crossing, bipoles/crossing/size=0.5] (3.5,-0.5-7);
	\draw (3.5,-0.5-7) to[crossing, bipoles/crossing/size=0.5] (3.5,0.5-7);
	\draw (3.5, 0.5-7) to[crossing, bipoles/crossing/size=0.5] (3.5,1.5-7);
	\draw (3.5, 1.5-7) to (3.5,1.5-7);
	\draw (3.5,2.0-1.77-6) node[npn, anchor=S] {};
	\draw (3.5, 2.5-7) to (3.5,-1.5-1);
	\draw (3.5, -1.5-1) to[crossing, bipoles/crossing/size=0.5] (3.5,-0.5-1);
	\draw (3.5,-0.5-1) to[crossing, bipoles/crossing/size=0.5] (3.5,0.5-1);
	\draw (3.5, 0.5-1) to[crossing, bipoles/crossing/size=0.5] (3.5,1.5-1);
	\draw (3.5, 1.5-1) to (3.5,2.0-1.5);
	\draw (3.5,2.0-1.77) node[npn, anchor=S] {};
	\draw (3.5, 2.0-0.5) to (3.5,4);

	% Circuit wires for B
	\draw (5.0, -10) to (5.0,-1.5-7);
	\draw (5.0,-1.5-7) to[crossing, bipoles/crossing/size=0.5] (5.0,-0.5-7);
	\draw (5.0,-0.5-7) to[crossing, bipoles/crossing/size=0.5] (5.0,0.5-7);
	\draw (5.0,2.0-1.77-7) node[npn, anchor=S] {};
	\draw (5.0, 2.5-7-1) to (5.0,-1.5-1);
	\draw (5.0, -1.5-1) to[crossing, bipoles/crossing/size=0.5] (5.0,-0.5-1);
	\draw (5.0,-0.5-1) to[crossing, bipoles/crossing/size=0.5] (5.0,0.5-1);
	\draw (5.0,2.0-1.77-1) node[npn, anchor=S] {};
	\draw (5.0, 2.0-0.5-1) to (5.0,3);

	% Circuit wires for C
	\draw (6.5, -10) to (6.5,-1.5-7);
	\draw (6.5,-1.5-7) to[crossing, bipoles/crossing/size=0.5] (6.5,-0.5-7);
	\draw (6.5,2.0-1.77-7-1) node[npn, anchor=S] {};
	\draw (6.5, 2.5-7-1-1) to (6.5,-1.5-1);
	\draw (6.5, -1.5-1) to[crossing, bipoles/crossing/size=0.5] (6.5,-0.5-1);
	\draw (6.5,2.0-1.77-1-1) node[npn, anchor=S] {};
	\draw (6.5, 2.0-0.5-1-1) to (6.5,3);

	% Circuit wires for D
	\draw (8.0, -10) to (8.0,-1.5-7);
	\draw (8.0,2.0-1.77-7-1-1) node[npn, anchor=S] {};
	\draw (8.0, 2.5-7-1-1-1) to (8.0,-1.5-1);
	\draw (8.0,2.0-1.77-1-1-1) node[npn, anchor=S] {};
	\draw (8.0, 2.0-0.5-1-1-1) to (8.0,3);

	% Circuit Completion
	\draw (3.5,3) to (8,3);
	\draw (8, -10) to (3.5, -10);
	\draw[purple] (-3.3, 3.3) rectangle (10.3, -10.3);

	% Power line
	\draw[dashed] (2, 4) -- (2, -11);

\end{circuitikz}

%% WIRE RECOVERY https://citeseerx.ist.psu.edu/document?repid=rep1&type=pdf&doi=3f948f763e4d77467a6bd6fc07e717870208.0d0 %%
%% http://tandfonline.com/doi/full/10.1080/01611194.2016.1257522 %%

%% note: I am not a historian, I dont read German. There are many protocls by which Enigma was used over varying time frames. There were many techniques attempted in breaking Enigma and many iterations of these techniques. The techiques discussed in this paper are intended to present a narrative leading to ultimately new results, but do not capture the full breadth and historical context surrounding the Enigma machine. The more I have studied this subject the more I have found that a truly complete analysis of the subject capturing both histoical, engineering, and matehmatical accuracy would require the combined efforts of many parties and would...%%

\chapter{The UK Bombe}

\section{Motivating Example}

\section{Changes to Enigma}

Starting in 1940, the German's enhanced the security of their 
key distribution. As discussed in CITE the \emph{Grundstellung} rotor 
position was sent along with the daily key and an operator chose a \emph{Spruchschlusse} to 
encode twice at the start of a message. Later iterations of this protocol removed the \emph{Grundstellung}
from key sheets.
\\\\These new key sheets contained the following columns
columns \emph{Tag/Datum}, \emph{Walzenlage}, \emph{Ringstellung}, \emph{Steckerverbindungen}, and \emph{Kenngruppen} 
\\\\Notice the removal of the \emph{Grundstellung} as well as the addition of the \emph{Kenngruppen}. The \emph{Kenngruppen} were a set of 
four trigrams used to identify which setting was being used to encode a message, this is particularly useful if trying to decode a message using a prior day's key. 
The operator would choose a trigram from the the \emph{Kenngruppen}, append two letters to the front of the trigram,
and this five letter combination (known as the \emph{Buchstabenkenngruppe}) would preceed the message being sent. If a message
was sent in multiple segments, multiple \emph{Buchstabenkenngruppe} were used to start each segment.
\\\\When sending a message the operator was to use the following protocol
\begin{enumerate}[I.]
\item The time at which the message was sent is listed
\item The number of parts which the message contained is listed
\item Which message part is being sent is listed
\item The length of the message part (not including \emph{Buchstabenkenngruppe}) is listed
\item A \emph{Grundstellung} rotor position is chosen and listed 
\item A \emph{Spruchschlüssel} rotor position is chosen and encoded using the \emph{Grundstellung}, this is listed
\item The \emph{Buchstabenkenngruppe} is listed
\item The message part encoded using the daily key and the \emph{Spruchschlüssel} position is listed
\end{enumerate}
It is clear that with this protocol, the Polish Bomba could no longer deduce the necessary 
details to decrypt enigma messages. All of the permutation information contained in the original key distribution 
protocol was removed and a new method needed to be derived for infering information about the daily key.
\section{Loops}
The removal of the double encoded \emph{Spruchschlüssel} does not mean that permutation information cannot be stored elsewhere in the message. 
For the sake of argument, let us say we knew that our encrypted message had plaintext encoding

\begin{center}
\begin{tikzpicture}[node distance=1cm, every node/.style={draw, circle, minimum height=0.1cm, minimum width=0.1cm}]

    % Centering the diagram
    \node (a1) [] {D};
    \node (a2) [right=0.1cm of a1] {Y};
    \node (a3) [right=0.1cm of a2] {Y};
    \node (a4) [right=0.1cm of a3] {Y};
    \node (a5) [right=0.1cm of a4] {Y};
    \node (a6) [right=0.1cm of a5] {Y};
    \node (a7) [right=0.1cm of a6] {X };
    \node (a8) [right=0.1cm of a7] {Y};
    \node (a9) [right=0.1cm of a8] {Y};
    \node (a10) [right=0.1cm of a9] {Y};
    \node (a11) [right=0.1cm of a10] {X};
    
    % Nodes for ciphertext
    \node (x1) [below=1cm of a1] {A};
    \node (x2) [below=1cm of a2] {B};
    \node (x3) [below=1cm of a3] {R};
    \node (x4) [below=1cm of a4] {A};
    \node (x5) [below=1cm of a5] {C};
    \node (x6) [below=1cm of a6] {A};
    \node (x7) [below=1cm of a7] {D};
    \node (x8) [below=1cm of a8] {A};
    \node (x9) [below=1cm of a9] {B};
    \node (x10) [below=1cm of a10] {R};
    \node (x11) [below=1cm of a11] {A};
    
    % Arrows for mapping
    \draw[->] (a1) -- (x1) node[midway, left, draw=none, fill=none] {1};
    \draw[->] (a2) -- (x2) node[midway, left, draw=none, fill=none] {2};
    \draw[->] (a3) -- (x3) node[midway, left, draw=none, fill=none] {3};
    \draw[->] (a4) -- (x4) node[midway, left, draw=none, fill=none] {4};
    \draw[->] (a5) -- (x5) node[midway, left, draw=none, fill=none] {5};
    \draw[->] (a6) -- (x6) node[midway, left, draw=none, fill=none] {6};
    \draw[->] (a7) -- (x7) node[midway, left, draw=none, fill=none] {7};
    \draw[->] (a8) -- (x8) node[midway, left, draw=none, fill=none] {8};
    \draw[->] (a9) -- (x9) node[midway, left, draw=none, fill=none] {9};
    \draw[->] (a10) -- (x10) node[midway, left, draw=none, fill=none] {10};
    \draw[->] (a11) -- (x11) node[midway, left, draw=none, fill=none] {11};
    
    \end{tikzpicture}
\end{center}

Where the top row is the ciphertext and the bottom is the plaintext, further, the number in each mapping indicates how many steps away we are from the rotor 
positions when we began encoding the message.

\begin{center}
    \begin{tikzpicture}[node distance=1cm, every node/.style={draw, circle, minimum height=0.1cm, minimum width=0.1cm}]
        \node (A) at (90:1.5) {$A$};
        \node (D) at (210:1.5) {$D$};
        \node (X) at (330:1.5) {$X$};
      
        \draw[->, bend right=45] (A) to node[midway, draw=none, above] {1} (D);
        \draw[->, bend right=45] (D) to node[midway, draw=none, above] {7} (X);
        \draw[->, bend right=45] (X) to node[midway, draw=none, above] {11} (A);
      \end{tikzpicture}
\end{center}

Recall
\begin{align*}
    E_1 &= P^{-1}\theta_1R_1^{-1}\theta_1^{-1}R_2^{-1}R_3^{-1}MR_3R_2\theta_1^{-1}R_1\theta_1P
    \\E_7 &= P^{-1}\theta_7R_1^{-1}\theta_7^{-1}R_2^{-1}R_3^{-1}MR_3R_2\theta_7^{-1}R_1\theta_7P
    \\E_{11} &= P^{-1}\theta_{11}R_1^{-1}\theta_{11}^{-1}R_2^{-1}R_3^{-1}MR_3R_2\theta_{11}^{-1}R_1\theta_{11}P
\end{align*}
Then it follows that our loop can be represented by 
\begin{align*}
    \sigma &= E_{11}\circ E_7 \circ E_{1}
\end{align*}
and we see that all the intermediate plugboard settings cancel out. Lets isolate the plugboard settings by letting 
$\overline{\sigma}$ represent $\sigma$ without the use of the plugboard for input and output, then 
\[
    \sigma = P^{-1}\overline{\sigma}P
\]
We have that
\begin{align*}
    \sigma(A) &= A
    \\\iff (P^{-1}\overline{\sigma}P)(A) &= A
    \\\iff \overline{\sigma}(P(A)) &= P(A)
\end{align*}

Suppose that our initial rotor position was correct, then certainly our $\overline{\sigma}$ is correct. We can make a hypothesis 
that $A$ is steckered to $K$ in the plugboard. Suppose we find that $\overline{\sigma}(K) \ne K$, then $\sigma(A)\ne A$ and our loop is broken, breaking our assumptions, thus $A$ must not be steckered to $K$.
But this will actually elimiate more hypotheses than just $A$ being steckered to $K$. We know that $\overline{\sigma}(K)$ is some letter which is not $K$. So we continue with a new hypothesis that $A$ is steckered to 
$\overline{\sigma}(K)$ and if we find $\overline{\sigma}(\overline{\sigma}(K)) \ne \overline{\sigma}(K)$, then we have further eliminated this possibility. 
Each new hypothesis suggests that $A$ is steckered to $\overline{\sigma}^{i}(K)$ which will be shown to be false if $\overline{\sigma}^{i+1}(K) \ne \overline{\sigma}^{i}(K)$. What if we find that 
$\overline{\sigma}^{i+1}(K) = \overline{\sigma}^i(K)$ at some point? This cannot happen since 
\begin{center}
        \begin{align*}
            &\overline{\sigma}^{i+1}(K) = \overline{\sigma}^i(K)
            \\\Rightarrow \text{ }&\overline{\sigma}^{-i}\circ\overline{\sigma}^{i+1}(K) = \overline\sigma^{-i}\circ\overline{\sigma}^i(K)
            \\\Rightarrow \text{ }&\overline{\sigma}(K) = K
        \end{align*}
\end{center}
which by supposition is false. Then we can continue in our hypotheses until we eventually reach a cycle where $\overline\sigma^i(K) = K$.
Then we gather a set of impossible steckerings, that is
\begin{align*}
    P(A) \notin \{\text{ }\overline{\sigma}^i(K)\text{ }\vert\text{ }i\in\mathbb{N}\}
\end{align*}
\\\\The notation we are using can be simplified significantly. The set $\{\text{ }\overline{\sigma}^i(K)\text{ }\vert\text{ }i\in\mathbb{N}\}$ is equivalent to the orbit of $K$ via the group action of the 
cyclic subgroup $\langle\overline{\sigma}\rangle$ which can be denoted $\langle\overline{\sigma}\rangle\cdot K$. 
\\\\We then have several cases 
\begin{enumerate}
    \item If $|\langle\overline{\sigma}\rangle\cdot K| = 26$, then $A$ cannot be steckered to anything which is clearly
    impossible, thus our rotor position must be incorrect. 
    \item If $|\langle\overline{\sigma}\rangle\cdot K| = 25$, then $A$ can only be steckered to the remaining letter \\$\{A,\dots,Z\} -
    \langle \overline{\sigma} \rangle\cdot K$
    \item If $|\langle\overline{\sigma}\rangle\cdot K| = 1$, in this case we must have intially had $\overline{\sigma}(K) = K$ so we have not
    eliminated any possibilities. 
\end{enumerate}

\section*{Motivating Example}

Suppose we knew the plaintext which had been enciphered into a particular enigma transmission.
Consider the following mapping,
\begin{center}
    \begin{tikzpicture}[node distance=1cm, every node/.style={draw, circle, minimum height=0.1cm, minimum width=0.1cm}]
    
        % Centering the diagram
        \node (a1) [] {D};
        \node (a2) [right=0.1cm of a1] {Y};
        \node (a3) [right=0.1cm of a2] {Y};
        \node (a4) [right=0.1cm of a3] {Y};
        \node (a5) [right=0.1cm of a4] {Y};
        \node (a6) [right=0.1cm of a5] {Y};
        \node (a7) [right=0.1cm of a6] {X };
        \node (a8) [right=0.1cm of a7] {Y};
        \node (a9) [right=0.1cm of a8] {Y};
        \node (a10) [right=0.1cm of a9] {Y};
        \node (a11) [right=0.1cm of a10] {X};
        
        % Nodes for ciphertext
        \node (x1) [below=1cm of a1] {A};
        \node (x2) [below=1cm of a2] {B};
        \node (x3) [below=1cm of a3] {R};
        \node (x4) [below=1cm of a4] {A};
        \node (x5) [below=1cm of a5] {C};
        \node (x6) [below=1cm of a6] {A};
        \node (x7) [below=1cm of a7] {D};
        \node (x8) [below=1cm of a8] {A};
        \node (x9) [below=1cm of a9] {B};
        \node (x10) [below=1cm of a10] {R};
        \node (x11) [below=1cm of a11] {A};
        
        % Arrows for mapping
        \draw[->] (a1) -- (x1) node[midway, left, draw=none, fill=none] {1};
        \draw[->] (a2) -- (x2) node[midway, left, draw=none, fill=none] {2};
        \draw[->] (a3) -- (x3) node[midway, left, draw=none, fill=none] {3};
        \draw[->] (a4) -- (x4) node[midway, left, draw=none, fill=none] {4};
        \draw[->] (a5) -- (x5) node[midway, left, draw=none, fill=none] {5};
        \draw[->] (a6) -- (x6) node[midway, left, draw=none, fill=none] {6};
        \draw[->] (a7) -- (x7) node[midway, left, draw=none, fill=none] {7};
        \draw[->] (a8) -- (x8) node[midway, left, draw=none, fill=none] {8};
        \draw[->] (a9) -- (x9) node[midway, left, draw=none, fill=none] {9};
        \draw[->] (a10) -- (x10) node[midway, left, draw=none, fill=none] {10};
        \draw[->] (a11) -- (x11) node[midway, left, draw=none, fill=none] {11};
        
        \end{tikzpicture}
    \end{center}
    where the top row indicates our enciphered message, the bottom row indicates the plaintext,
    and then indices on the arrows indicate how many steps forward our enigma machine has moved while enciphering this message.
    Our goal is to determine which enigma settings were used to encipher the message.  In order to achieve this, 
    we will examine which settings maintain the relationships between the enciphered and plaintext letters. 
    \\\\For example, any setting which maintains the above pairing must encipher $A$ to $D$ from the first position of the machine, then at 
    the seventh position, it must encipher $D$ to $X$, and at the eleventh position $X$ must be enciphered back to $A$. It follows that if we had 
    three enigma machines connected in series, with an offset of 1, 7, and 11, from our initial position, then inputting $A$ on the first machine would result in an ouput of $A$ on the
    third machine. We visualize this loop as follows 
    \begin{center}
        \begin{tikzpicture}[node distance=1cm, every node/.style={draw, circle, minimum height=0.1cm, minimum width=0.1cm}]
            \node (A) at (90:1.5) {$A$};
            \node (D) at (210:1.5) {$D$};
            \node (X) at (330:1.5) {$X$};
          
            \draw[<->, bend right=45] (A) to node[midway, draw=none, above] {1} (D);
            \draw[<->, bend right=45] (D) to node[midway, draw=none, above] {7} (X);
            \draw[<->, bend right=45] (X) to node[midway, draw=none, above] {11} (A);
          \end{tikzpicture}
    \end{center}
    To express this mathematically we denote the permutation represented by the enigma at 
    position $i$ as $\sigma_i$. Since these each use the same plugboard we will also note the 
    enigma at position $i$ not using the plugboard as $\overline{\sigma_i}$, that is $\sigma_i = P\overline{\sigma_i}P$.
    Then our loop is expressed by the fact that $\sigma_{11}\circ\sigma_7\circ\sigma_1$ has a fixed point at $A$. 
    We also note that the intermediate plugboard settings cancel out, that is 
    \begin{center}
        \begin{align*}
            \sigma_{1}\circ\sigma_7\circ\sigma_{11} &= P\overline{\sigma_{1}}P\circ P\overline{\sigma_7}P\circ P\overline{\sigma_{11}}P
            \\&= P\circ \overline{\sigma_{1}}\circ\overline{\sigma_7}\circ\overline{\sigma_{11}}\circ P
        \end{align*}
    \end{center}
    We will condense this notation by defining
    \begin{center}
        $\sigma \coloneq \sigma_{1}\circ\sigma_7\circ\sigma_{11}$
    \end{center}
    and 
    \begin{center}
        $\overline{\sigma} \coloneq \overline{\sigma_{1}}\circ\overline{\sigma_7}\circ\overline{\sigma_{11}}$
    \end{center}
    And thus we have shown $\sigma = P\overline{\sigma}P$.
    \\\\Let us hypothesize that $A$ is steckered in the plugboard $\alpha$ -- that is, $P(A) = \alpha$. It then follows that 
    \begin{center}
        \begin{align*}
            \overline{\sigma}^i(\alpha) &= P\circ\sigma^i\circ P(\alpha)
            \\&= P\circ\sigma^i(A)
            \\&= P(A)
        \end{align*}
    \end{center} 
    and so we derive 
    \begin{center}
        $P(A) = \alpha \Rightarrow P(A) = \overline{\sigma^i}(\alpha)\text{ }\forall\text{ }i\in\mathbb{N}$
    \end{center}
    Then we have that $A$ must be steckered to all values in the set $\{\overline{\sigma}^i(\alpha)\text{ }\vert\text{ }i\in\mathbb{N}\}$. 
    We note that this set is that orbit of the element $\alpha$ under the group action of the subgroup $\langle\overline{\sigma}\rangle$ -- that is, 
    $\langle\overline{\sigma}\rangle\cdot\alpha$. 
    \\\\By construction of the enigma machine, $A$ cannot be steckered to more than one value at a time, so if $|\langle\overline{\sigma}\rangle\cdot\alpha| > 1$ our initial
    hypotheses that $P(A) = \alpha$ must have been incorrect. Further, the above arugment also illustrates that $A$ cannot be steckered to any element in the orbit of $\alpha$ since 
    we would similarly find that the orbit of that element was not a singleton. Then we now have 
    \begin{center}
        $|\langle\overline{\sigma}\rangle\cdot\alpha| > 1 \Rightarrow P(A) \notin \langle\overline{\sigma}\rangle\cdot\alpha$
    \end{center} 
    thus eliminating several elements that $A$ could be steckered to. 
    \\\\In fact, if $|\langle\overline{\sigma}\rangle\cdot\alpha| = 26$
    then $A$ cannot be steckered to \emph{any} element. Of course, by construction, $A$ must be steckered to \emph{some} element, so it must be the case
    that this rotor position is impossible since no plugboard setting could produce the loop relationship in our ciphertext-plaintext pairing. 
    \\\\This is the 
    key idea in the construction of the Bombe. At a high level, we encode the structure of the ciphertext-plaintext pairings, we input a hypothesis that a particular 
    letter (e.g $A$) is steckered to another (e.g. $\alpha$), and we electrically produce the elements that must also be steckered to our test letter 
    (e.g. $\langle\overline{\sigma}\rangle\cdot\alpha$) if our hypothesis were to be true. If we find that this production generates a set of all letters and thus disqualifies this rotor position from producing
    the plaintext we observed, then we can continue on to the next rotor position until we have no contradictory results from our hypothesis.
    \\\\Before describing the function of the machine more concretely, we will first examine how we as humans can examine such permutations and deduce hypotheses quickly. 
    
    \section{The Bombe}

    In the section, we outline the construction of a rudementary Bombe. Our goal is to construct a machine using the above insight to quickly elimate a rotor position based on contradictory hypotheses. For clarity sake, we will 
    imagine that our Engima machine operates on an alphabet of only four letters $\{A, B, C, D\}$.  
    We must shift from the above mathematical construction to an electrical one. We first abstract the idea of the Engima machine and do away with the input output system of keys and lamp lights.
    Rather, we imagine the Enigma machine as black-box that takes in 4 cables encoding, via current, each of the 4 letters, and outputs currents on the corresponding letter after applying the Enigma permutation.
    Imagine the following set of wires encoding a possible Enigma permutation we will denoted $\overline\sigma_\alpha$ (the bar indicates we are not yet considering the plugboard).
    
    \tikzset{big box/.style={draw, minimum width=5cm, minimum height=4cm},
    small box/.style={draw, minimum width=1cm, minimum height=1cm}}

    \begin{center}
        \begin{tikzpicture}[thick]
            % Draw the box
            \draw[fill=lightgray] (2,-1.5) rectangle (4,2.5) node[midway] {};
            
            \node at (3, -2) {$\overline\sigma_\alpha$};

            % Draw the wires entering the box
            \draw[-] (0, 2) -- (2, 2) node[midway, above] {A};
            \draw[-] (0, 1) -- (2, 1) node[midway, above] {B};
            \draw[-] (0, 0) -- (2, 0) node[midway, above] {C};
            \draw[-] (0,-1) -- (2,-1) node[midway, above] {D};

            % Draw the wires exiting the box with crossed mappings
            \draw[-] (4, 2) -- (6,2) node[midway, above] {A};
            \draw[-] (4, 1) -- (6, 1) node[midway, above] {B};
            \draw[-] (4, 0) -- (6, 0) node[midway, above] {C};
            \draw[-] (4,-1) -- (6, -1) node[midway, above] {D};

            % Draw the lines inside the box to represent the mapping
            \draw[-] (2, 2) -- (4,-1);
            \draw[-] (2, 1) -- (4, 0);
            \draw[-] (2, 0) -- (4, 1);
            \draw[-] (2,-1) -- (4, 2);

        \end{tikzpicture} 
    \end{center}
    A couple quick notes about this abstraction. First as these lines are simply wires, current can flow in either direction, left-to-right, or right-to-left. 
    Second, we can apply current to multiple wires concurrently, for example, applying current at $A$ and $C$ will cause $D$ and $B$ to be live on the other side of the machine.
    \\\\As described in our motivating example, we may want to connect Engima permutations in series to capture an underlying relationship between a ciphertext-plaintext pairing. 
    Suppose analysis found that we have three Enigma permutations represented by $\overline\sigma_\alpha, \overline\sigma_\beta$, and $\overline\sigma_\gamma$ which form a loop having a fixed point at $A$. 
    \begin{center}
    \begin{tikzpicture}[thick, scale=0.75, every node/.style={scale=0.7}]
        % Define the positions for the three diagrams in a circle
        \foreach \x/\y/\label/\mapping in {0/4/{$\alpha$}/ {(4,-1), (4,0), (4,1), (4,2)},
        -4/-2.5/{$\beta$}/ {(4,-1), (4,0), (4,1), (4,2)},
        4/-2.5/{$\gamma$}/ {(4,-1), (4,0), (4,1), (4,2)}} {
            \begin{scope}[shift={(\x,\y)}]
                % Draw the box
                \draw[fill=lightgray] (2,-1.5) rectangle (4,2.5) node[midway] {};
            
                % Label the box with sigma below
                \node at (3, -2) {$\overline\sigma_{\text{A}}$};
    
                % Draw the wires entering the box
                \draw[-] (0, 2) -- (2, 2) node[midway, above] {A};
                \draw[-] (0, 1) -- (2, 1) node[midway, above] {B};
                \draw[-] (0, 0) -- (2, 0) node[midway, above] {C};
                \draw[-] (0,-1) -- (2,-1) node[midway, above] {D};
    
                % Draw the wires exiting the box with crossed mappings
                \draw[-] (4, 2) -- (6,2) node[midway, above] {A};
                \draw[-] (4, 1) -- (6, 1) node[midway, above] {B};
                \draw[-] (4, 0) -- (6, 0) node[midway, above] {C};
                \draw[-] (4,-1) -- (6, -1) node[midway, above] {D};
    
                \foreach \endX/\endY in \mapping {
                    % Extract start and end coordinates from the mapping list
                    \pgfmathsetmacro{\startX}{2}
                    \pgfmathsetmacro{\startY}{2 - \i + 1}
                    \draw[-] (\startX, \startY) -- (\endX, \endY);
                }
                % Draw the lines inside the box to represent the mapping
            \end{scope}
            \draw[-] (0, 6) to[out=180, in=180] (-4, -0.5); % Connect A
            \draw[-] (0, 5) to[out=180, in=180] (-4, -1.5); % Connect B
            \draw[-] (0, 4) to[out=180, in=180] (-4,-2.5); % Connect C
            \draw[-] (0, 3) to[out=180, in=180] (-4,-3.5); % Connect D

            \draw[-] (0, -0.5) -- (6, -0.5); % Connect A
            \draw[-] (0, -1.5) -- (6, -1.5); % Connect A
            \draw[-] (0, -2.5) -- (6, -2.5); % Connect A
            \draw[-] (0, -3.5) -- (6, -3.5); % Connect A

            \draw[-] (6, 6) to[out=360, in=360] (10, -0.5); % Connect A
            \draw[-] (6, 5) to[out=360, in=360] (10, -1.5); % Connect B
            \draw[-] (6, 4) to[out=360, in=360] (10,-2.5); % Connect C
            \draw[-] (6, 3) to[out=360, in=360] (10,-3.5); % Connect D
        }
    \end{tikzpicture}
\end{center}
    \section{Cycles}
    Another way of viewing the orbit of the element $\alpha$ under the action by $\langle\overline\sigma\rangle$ in our motivating example, is to consider the cycle in $\overline\sigma$ containing $\alpha$.
    Suppose our alphabet had six letters $\{\alpha, \beta, \gamma, \delta, \epsilon, \zeta\}$. We may find that in our cycle notation we can write $\overline\sigma$ as 
    \begin{center}
        $(\alpha\beta\gamma)(\delta\epsilon)(\zeta)$
    \end{center}
    From this, we can determine that had our hypotheses been that $P(A) = \alpha$ we would find that $P(A) = \beta$ and $P(A) = \gamma$. 
    Similarly, had we hypothesized that $P(A) = \delta$ we would have found that $P(A) = \epsilon$. This is to say, if we hypothesize that $P(A)$ is mapped to a particular
    element, then it must also be mapped to all elements in the cycle containing that element. 

  
\chapter{Stops}
The Turing-Welchman Bombe was remarkably adept at computing daily
keys in a tractable manner. In optimal conditions,
with a well-crafted and fortunate menu, the Bombe could run in as
little as 20 minutes and produce exactly the daily key needed to
decrypt messages.
To approach these optimal conditions, we must consider what it means
for one menu to be stronger than another. In this chapter, we will
explore the relationship
between menus and the number of stops the Bombe is expected to
encounter. Equipped with this information, a cryptanalyst would be
able to select the menu
that yields the fewest stops. Each stop consumed precious
time as operators needed to verify whether they corresponded to a
valid key. Therefore, menu selection becomes an integral factor
in reducing the time between intercepting transmissions and
recovering the key for that day. Even a reduction of minutes could
provide the slight edge that the Allies needed to preempt an attack.
Thus, developing an accurate model by which we can correlate menus to
their expected number of stops is a matter of strategic urgency.

\section{Turing's Model}
We begin by describing the model for the expected number of stops given by Turing himself in the Prof's Book. Turing approaches this problem not by computing the expected number of stops itself, but instead by computing the expected number of {\bf{normal stops}} over all steckering hypothesis. These normal stops are described by Turing as ``positions at which by altering the point at which the current enters the diagonal board, one can make 25 relays close.'' In the case of a single loop in our menu, this is equivalent to the statement that the resulting loop in the Bombe has a singleton cycle. If we apply current to this singleton cycle all the remaining relays will close. For now we will ignore the impact of the diagonal board.
\\\\Turing considered a menu in which no loops occured which he called a {\bf{web}}. In this case every position of the Bombe and every initial steckering hypothesis would create a normal stop as there is no feedback necessary to electrify any additional wires on a given cable. Turing considers not only the $26^3$ possible rotor positions of the Bombe, but also the $26$ initial steckering hypotheses we could input. In our case, all steckering hypotheses and all rotor configurations produce a normal stop so we get $26^4$ total normal stops over all steckering hypotheses. In our simplified model with only $4$ characters this may look as follows
\begin{center}
	\begin{tikzpicture}[thick, scale=0.4, every node/.style={scale=0.7}]

		% Draw the wires entering the box
		\draw[-] (0, 2) -- (2, 2) node[midway, above] {\texttt{a}};
		\draw[-] (0, 1) -- (2, 1) node[midway, above] {\texttt{b}};
		\draw[-] (0, 0) -- (2, 0) node[midway, above] {\texttt{c}};
		\draw[-] (0,-1) -- (2,-1) node[midway, above] {\texttt{d}};

		% Draw the wires exiting the box with crossed mappings
		\draw[-] (4, 2) -- (6,2) node[midway, above] {\texttt{a}};
		\draw[-] (4, 1) -- (6, 1) node[midway, above] {\texttt{b}};
		\draw[-] (4, 0) -- (6, 0) node[midway, above] {\texttt{c}};
		\draw[-] (4,-1) -- (6, -1) node[midway, above] {\texttt{d}};

		\draw[-] (0-3, 2-7) to[out=180, in=180] (0, 2) node[midway, above] {};
		\draw[-] (0-3, 1-7) to[out=180, in=180] (0, 1) node[midway, above] {};
		\draw[-] (0-3, 0-7) to[out=180, in=180] (0, 0) node[midway, above] {};
		\draw[-] (0-3, -1-7) to[out=180, in=180] (0, -1)
		node[midway, above] {};

		\draw[-] (6+3, 2-7) to[out=360, in=360] (6, 2) node[midway, above] {};
		\draw[-] (6+3, 1-7) to[out=360, in=360] (6, 1) node[midway, above] {};
		\draw[-] (6+3, 0-7) to[out=360, in=360] (6, 0) node[midway, above] {};
		\draw[-] (6+3, -1-7) to[out=360, in=360] (6, -1)
		node[midway, above] {};

		% Draw the wires entering the box
		\draw[-] (0-3, 2-7) -- (2, 2-7) node[midway, above] {\texttt{a}};
		\draw[-] (0-3, 1-7) -- (2, 1-7) node[midway, above] {\texttt{b}};
		\draw[-] (0-3, 0-7) -- (2, 0-7) node[midway, above] {\texttt{c}};
		\draw[-] (0-3,-1-7) -- (2,-1-7) node[midway, above] {\texttt{d}};

		% Draw the wires exiting the box
		\draw[-] (2+2, 2-7) -- (4+5, 2-7) node[midway, above] {\texttt{a}};
		\draw[-] (2+2, 1-7) -- (4+5, 1-7) node[midway, above] {\texttt{b}};
		\draw[-] (2+2, 0-7) -- (4+5, 0-7) node[midway, above] {\texttt{c}};
		\draw[-] (2+2,-1-7) -- (4+5,-1-7) node[midway, above] {\texttt{d}};

		% Draw the lines inside the box to represent the mapping
		\draw[fill=lightgray] (2,-1.5-7) rectangle (4,2.5-7) node[midway] {};

		\node at (3, -2-7) {$\overline\pi$};


	\end{tikzpicture}
\end{center}
\noindent Turing then considered the effect of adding an edge to our menu which would have the effect of forming a loop, he called such an edge a {\bf{chain-closing constatation}}. He wanted to deduce the likelihood that adding such an edge turns our normal stops into anything other than a normal stop. We will denote this chain-closing edge as $s$. Our diagram would then be
\begin{center}
	\begin{tikzpicture}[thick, scale=0.4, every node/.style={scale=0.7}]

		% Draw the lines inside the box to represent the mapping
		\draw[fill=lightgray] (2,-1.5) rectangle (4,2.5) node[midway] {};

		\node at (3, -2) {$s$};

		% Draw the wires entering the box
		\draw[-] (0, 2) -- (2, 2) node[midway, above] {\texttt{a}};
		\draw[-] (0, 1) -- (2, 1) node[midway, above] {\texttt{b}};
		\draw[-] (0, 0) -- (2, 0) node[midway, above] {\texttt{c}};
		\draw[-] (0,-1) -- (2,-1) node[midway, above] {\texttt{d}};

		% Draw the wires exiting the box with crossed mappings
		\draw[-] (4, 2) -- (6,2) node[midway, above] {\texttt{a}};
		\draw[-] (4, 1) -- (6, 1) node[midway, above] {\texttt{b}};
		\draw[-] (4, 0) -- (6, 0) node[midway, above] {\texttt{c}};
		\draw[-] (4,-1) -- (6, -1) node[midway, above] {\texttt{d}};

		\draw[-] (0-3, 2-7) to[out=180, in=180] (0, 2) node[midway, above] {};
		\draw[-] (0-3, 1-7) to[out=180, in=180] (0, 1) node[midway, above] {};
		\draw[-] (0-3, 0-7) to[out=180, in=180] (0, 0) node[midway, above] {};
		\draw[-] (0-3, -1-7) to[out=180, in=180] (0, -1)
		node[midway, above] {};

		\draw[-] (6+3, 2-7) to[out=360, in=360] (6, 2) node[midway, above] {};
		\draw[-] (6+3, 1-7) to[out=360, in=360] (6, 1) node[midway, above] {};
		\draw[-] (6+3, 0-7) to[out=360, in=360] (6, 0) node[midway, above] {};
		\draw[-] (6+3, -1-7) to[out=360, in=360] (6, -1)
		node[midway, above] {};

		% Draw the wires entering the box
		\draw[-] (0-3, 2-7) -- (2, 2-7) node[midway, above] {\texttt{a}};
		\draw[-] (0-3, 1-7) -- (2, 1-7) node[midway, above] {\texttt{b}};
		\draw[-] (0-3, 0-7) -- (2, 0-7) node[midway, above] {\texttt{c}};
		\draw[-] (0-3,-1-7) -- (2,-1-7) node[midway, above] {\texttt{d}};

		% Draw the wires exiting the box
		\draw[-] (2+2, 2-7) -- (4+5, 2-7) node[midway, above] {\texttt{a}};
		\draw[-] (2+2, 1-7) -- (4+5, 1-7) node[midway, above] {\texttt{b}};
		\draw[-] (2+2, 0-7) -- (4+5, 0-7) node[midway, above] {\texttt{c}};
		\draw[-] (2+2,-1-7) -- (4+5,-1-7) node[midway, above] {\texttt{d}};

		% Draw the lines inside the box to represent the mapping
		\draw[fill=lightgray] (2,-1.5-7) rectangle (4,2.5-7) node[midway] {};

		\node at (3, -2-7) {$\overline\pi$};


	\end{tikzpicture}
\end{center}
\noindent Suppose for a particular steckering hypothesis $S(x) = y$, the electrification of our hypothesis wire produces a normal stop. What is the probability that by adding the permutation $s$ we are now no longer in the situation of a normal stop?
\\\\Given that by supposition electrifying wire $xy$ through $\overline\pi$ has only a single live wire, the permutation $s$ need only connect the live wire to any of the $3$ remaining non-electrified wires to arrive at anything other than a normal stop. Thus there is $\frac{3}{4}$ chance that $s\circ\overline\pi$ is no longer a normal stop. Conversely, there is a $\frac{1}{4}$ chance that $s$ fails to remove our normal stop.
\\\\In the case of a $26$ character alphabet the same logic follows and we have a $\frac{1}{26}$ chance that a chain-closing constatations fails to remove a normal stop. Given that we originally expected $26^4$ normal stops over all steckering hypotheses without any closures, we would now expect that adding a chain-closing constatation to our menu would have $\frac{1}{26}$ as many normal stops. Thus adding one closure to our menu now brings the expected number of normal stops over all steckering hypotheses down to $26^{4-1} = 26^3$.
\\\\Turing extends this argument to explain that for a menu with $c$ loops (which he called {\bf{closures}}) we would expect $26^{4-c}$ normal stops over all steckering hypotheses.
\subsection{Dixon's Theorem}
A natural question arises - why should the number of normal stops over all steckering hypotheses approximate the number of stops? First, we are only considering a subset of possible stops so we might expect our approximation to be poor. Second, we are considering these stops over all steckering hypotheses so we might expect our approximation to be around $26$ times greater than the actual answer. Yet, testing Turing's formula against the actual Bombe, in many cases we find that his approximation of $26^{4-c}$ falls roughly in line with the actual number of stops. To answer this question we must show the following
\begin{enumerate}[(I)]
	\item The number of normal stops approximates the number of stops
	\item The consideration of all $26$ steckering hypotheses for each rotor position averages out in such a way that we get only $1$ steckering hypothesis per each rotor position thus removing any double counting of a particular rotor position.
\end{enumerate}
\subsubsection{Transitive Subgroups}
\noindent To show these two points, we require additional mathematical tools. We first need to translate our mechanical understanding of the Bombe's electrifcation of wires into a mathematical one. For a single loop this is quite trivial. If the permutation representing our loop is $\overline\pi$, then the number of normal stops over all steckering hypotheses is just the number of singleton cycles in $\overline\pi$'s cycle decomposition.
\\\\Once we add in additional closures this becomes slightly more complicated. Suppose we have two closures, with each seperate closure being represented by the permutations $\overline\pi$ and $\overline\delta$ respectively.
\begin{figure}[H]
	\begin{center}
		\scalebox{0.8}{
			\begin{tikzpicture}[thick, scale=0.4, every node/.style={scale=0.7}]
				% Draw the box
				\draw[fill=lightgray] (0+5, 0+2) rectangle (2+5,4+2) node[midway] {};
				\node at (0+5+1, 0+2-0.5) {$\overline\pi_3$};

				% Draw the wires exiting the box
				\draw[-] (7, 5.5) -- (8,5.5) node[midway, above] {\texttt{a}};
				\draw[-] (7, 4.5) -- (8,4.5) node[midway, above] {\texttt{b}};
				\draw[-] (7, 3.5) -- (8, 3.5) node[midway, above] {\texttt{c}};
				\draw[-] (7,2.5) -- (8, 2.5) node[midway, above] {\texttt{d}};

				% Draw the wires entering the box
				\draw[-] (4, 5.5) -- (5,5.5) node[midway, above] {};
				\draw[-] (4, 4.5) -- (5,4.5) node[midway, above] {};
				\draw[-] (4, 3.5) -- (5, 3.5) node[midway, above] {};
				\draw[-] (4,2.5) -- (5, 2.5) node[midway, above] {};

				\draw[fill=lightgray] (0-5,0+2) rectangle (2-5,4+2) node[midway] {};
				\node at (0-5+1, 0+2-0.5) {$\overline\pi_1$};

				% Draw the wires entering the box
				\draw[-] (-13+7, 5.5) -- (-13+8,5.5) node[midway, above] {\texttt{a}};
				\draw[-] (-13+7, 4.5) -- (-13+8,4.5) node[midway, above] {\texttt{b}};
				\draw[-] (-13+7, 3.5) -- (-13+8, 3.5) node[midway, above] {\texttt{c}};
				\draw[-] (-13+7,2.5) -- (-13+8, 2.5) node[midway, above] {\texttt{d}};

				% Draw the wires exiting the box
				\draw[-] (-13+7+3, 5.5) -- (-13+8+3,5.5) node[midway, above] {};
				\draw[-] (-13+7+3, 4.5) -- (-13+8+3,4.5) node[midway, above] {};
				\draw[-] (-13+7+3, 3.5) -- (-13+8+3, 3.5) node[midway, above] {};
				\draw[-] (-13+7+3,2.5) -- (-13+8+3, 2.5) node[midway, above] {};

				\draw[fill=lightgray] (0,0+10) rectangle (2,4+10) node[midway] {};
				\node at (0+1, 0+10-0.5) {$\overline\pi_2$};

				% Draw the wires entering the box
				\draw[-] (-1, 3+10.5) -- (0, 3+10.5) node[midway, above] {\texttt{a}};
				\draw[-] (-1, 2+10.5) -- (0, 2+10.5) node[midway, above] {\texttt{b}};
				\draw[-] (-1, 1+10.5) -- (0, 1+10.5) node[midway, above] {\texttt{c}};
				\draw[-] (-1,0+10.5) -- (0,0+10.5) node[midway, above] {\texttt{d}};

				% Draw the wires exiting the box
				\draw[-] (3+-1, 3+10.5) -- (3+0, 3+10.5) node[midway, above]
				{\texttt{a}};
				\draw[-] (3+-1, 2+10.5) -- (3+0, 2+10.5) node[midway, above]
				{\texttt{b}};
				\draw[-] (3+-1, 1+10.5) -- (3+0, 1+10.5) node[midway, above]
				{\texttt{c}};
				\draw[-] (3+-1,0+10.5) -- (3+0,0+10.5) node[midway, above] {\texttt{d}};

				% Draw wires from 2 to 1
				\draw[-] (-6, 5.5) to[out=180, in=180] (-1, 13.5)
				node[midway, above] {};
				\draw[-] (-6, 4.5) to[out=180, in=180] (-1, 12.5)
				node[midway, above] {};
				\draw[-] (-6, 3.5) to[out=180, in=180] (-1, 11.5)
				node[midway, above] {};
				\draw[-] (-6, 2.5) to[out=180, in=180] (-1, 10.5)
				node[midway, above] {};

				% Draw wires from 3 to 2
				\draw[-] (8, 5.5) to[out=360, in=360] (3, 13.5) node[midway, above] {};
				\draw[-] (8, 4.5) to[out=360, in=360] (3, 12.5) node[midway, above] {};
				\draw[-] (8, 3.5) to[out=360, in=360] (3, 11.5) node[midway, above] {};
				\draw[-] (8, 2.5) to[out=360, in=360] (3, 10.5)
				node[midway, above] {};

				% Midway wires
				\draw[-] (0.5, 1) -- (1.5, 1) node[midway, above] {\texttt{a}};
				\draw[-] (0.5, 0) -- (1.5, 0) node[midway, above] {\texttt{b}};
				\draw[-] (0.5, -1) -- (1.5, -1) node[midway, above] {\texttt{c}};
				\draw[-] (0.5,-2) -- (1.5,-2) node[midway, above] {\texttt{d}};

				% Draw wires from 1 to midway
				\draw[-] (-2, 5.5) to[out=360, in=180] (0.5, 1) node[midway, above] {};
				\draw[-] (-2, 4.5) to[out=360, in=180] (0.5, 0) node[midway, above] {};
				\draw[-] (-2, 3.5) to[out=360, in=180] (0.5, -1) node[midway, above] {};
				\draw[-] (-2, 2.5) to[out=360, in=180] (0.5, -2) node[midway, above] {};

				% Draw wires from 2 to midway
				\draw[-] (4, 5.5) to[out=180, in=360] (1.5, 1) node[midway, above] {};
				\draw[-] (4, 4.5) to[out=180, in=360] (1.5, 0) node[midway, above] {};
				\draw[-] (4, 3.5) to[out=180, in=360] (1.5, -1) node[midway, above] {};
				\draw[-] (4, 2.5) to[out=180, in=360] (1.5, -2) node[midway, above] {};

				\draw[fill=lightgray] (0,0-15) rectangle (2,4-15) node[midway] {};
				\node at (0+1, 0+10-0.5-25) {$\overline\delta_2$};

				% Draw the wires entering the box
				\draw[-] (-1, 3+10.5-25) -- (0, 3+10.5-25) node[midway,
					above] {\texttt{a}};
				\draw[-] (-1, 2+10.5-25) -- (0, 2+10.5-25) node[midway,
					above] {\texttt{b}};
				\draw[-] (-1, 1+10.5-25) -- (0, 1+10.5-25) node[midway,
					above] {\texttt{c}};
				\draw[-] (-1,0+10.5-25) -- (0,0+10.5-25) node[midway, above]
				{\texttt{d}};

				% Draw the wires exiting the box
				\draw[-] (3+-1, 3+10.5-25) -- (3+0, 3+10.5-25) node[midway,
					above] {\texttt{a}};
				\draw[-] (3+-1, 2+10.5-25) -- (3+0, 2+10.5-25) node[midway,
					above] {\texttt{b}};
				\draw[-] (3+-1, 1+10.5-25) -- (3+0, 1+10.5-25) node[midway,
					above] {\texttt{c}};
				\draw[-] (3+-1,0+10.5-25) -- (3+0,0+10.5-25) node[midway,
					above] {\texttt{d}};

				\draw[fill=lightgray] (0-5,0+2-9) rectangle (2-5,4+2-9) node[midway] {};
				\node at (0-5+1, 0+2-0.5-9) {$\overline\delta_1$};

				% Draw the wires entering the box
				\draw[-] (-13+7, 5.5-9) -- (-13+8,5.5-9) node[midway, above]
				{\texttt{a}};
				\draw[-] (-13+7, 4.5-9) -- (-13+8,4.5-9) node[midway, above]
				{\texttt{b}};
				\draw[-] (-13+7, 3.5-9) -- (-13+8, 3.5-9) node[midway, above]
				{\texttt{c}};
				\draw[-] (-13+7,2.5-9) -- (-13+8, 2.5-9) node[midway, above]
				{\texttt{d}};

				% Draw the wires exiting the box
				\draw[-] (-13+7+3, 5.5-9) -- (-13+8+3,5.5-9) node[midway, above] {};
				\draw[-] (-13+7+3, 4.5-9) -- (-13+8+3,4.5-9) node[midway, above] {};
				\draw[-] (-13+7+3, 3.5-9) -- (-13+8+3, 3.5-9) node[midway, above] {};
				\draw[-] (-13+7+3,2.5-9) -- (-13+8+3, 2.5-9) node[midway, above] {};

				% Draw the box
				\draw[fill=lightgray] (0+5, 0+2-9) rectangle (2+5,4+2-9)
				node[midway] {};
				\node at (0+5+1, 0+2-0.5-9) {$\overline\delta_3$};

				% Draw the wires exiting the box
				\draw[-] (7, 5.5-9) -- (8,5.5-9) node[midway, above] {\texttt{a}};
				\draw[-] (7, 4.5-9) -- (8,4.5-9) node[midway, above] {\texttt{b}};
				\draw[-] (7, 3.5-9) -- (8, 3.5-9) node[midway, above] {\texttt{c}};
				\draw[-] (7,2.5-9) -- (8, 2.5-9) node[midway, above] {\texttt{d}};

				% Draw the wires entering the box
				\draw[-] (4, 5.5-9) -- (5,5.5-9) node[midway, above] {};
				\draw[-] (4, 4.5-9) -- (5,4.5-9) node[midway, above] {};
				\draw[-] (4, 3.5-9) -- (5, 3.5-9) node[midway, above] {};
				\draw[-] (4,2.5-9) -- (5, 2.5-9) node[midway, above] {};

				% Draw wires from 4 to 5
				\draw[-] (-6, 5.5-9) to[out=180, in=180] (-1, 13.5-25)
				node[midway, above] {};
				\draw[-] (-6, 4.5-9) to[out=180, in=180] (-1, 12.5-25)
				node[midway, above] {};
				\draw[-] (-6, 3.5-9) to[out=180, in=180] (-1, 11.5-25)
				node[midway, above] {};
				\draw[-] (-6, 2.5-9) to[out=180, in=180] (-1, 10.5-25)
				node[midway, above] {};

				% Draw wires from 5 to 6
				\draw[-] (8, 5.5-9) to[out=360, in=360] (3, 13.5-25)
				node[midway, above] {};
				\draw[-] (8, 4.5-9) to[out=360, in=360] (3, 12.5-25)
				node[midway, above] {};
				\draw[-] (8, 3.5-9) to[out=360, in=360] (3, 11.5-25)
				node[midway, above] {};
				\draw[-] (8, 2.5-9) to[out=360, in=360] (3, 10.5-25)
				node[midway, above] {};

				% Draw wires from 4 to midway
				\draw[-] (-2, 5.5-9) to[out=360, in=180] (0.5, 1)
				node[midway, above] {};
				\draw[-] (-2, 4.5-9) to[out=360, in=180] (0.5, 0)
				node[midway, above] {};
				\draw[-] (-2, 3.5-9) to[out=360, in=180] (0.5, -1)
				node[midway, above] {};
				\draw[-] (-2, 2.5-9) to[out=360, in=180] (0.5, -2)
				node[midway, above] {};

				% Draw wires from 2 to midway
				\draw[-] (4, 5.5-9) to[out=180, in=360] (1.5, 1) node[midway, above] {};
				\draw[-] (4, 4.5-9) to[out=180, in=360] (1.5, 0) node[midway, above] {};
				\draw[-] (4, 3.5-9) to[out=180, in=360] (1.5, -1)
				node[midway, above] {};
				\draw[-] (4, 2.5-9) to[out=180, in=360] (1.5, -2)
				node[midway, above] {};
				\node[draw,circle] at (-2, -0.5) {\texttt{A}};
				\node[draw,circle] at (-7, -11) {\texttt{B}};
				\node[draw,circle] at (-7, -11+21) {\texttt{B}};
				\node[draw,circle] at (-7+16, -11) {\texttt{C}};
				\node[draw,circle] at (-7+16, -11+21) {\texttt{C}};

			\end{tikzpicture}
		}
	\end{center}
\end{figure}
\noindent How can we determine which wires are connected to which in this complex arrangment of permutations? To answer this we introduce
\begin{definition}
	The subgroup of $S_n$ generated by two permutations $\sigma$, $\tau$ $\in S_n$ is
	\begin{center}
		$\langle\sigma,\tau\rangle \coloneq \{\sigma^{a_1}\tau^{b_1}\dots\sigma^{a_k}\tau^{b_k}\text{ }|\text{ }k\in\mathbb{N},\text{ }a_i,b_i\in\mathbb{Z}\}$
	\end{center}
\end{definition}
\noindent This is the set of all group elements obtained by applying finite compositions of $\sigma$, $\tau$, and their inverses.
\\\\Consider how this relates to our two loop Bombe arrangment with $\overline\pi$ and $\overline\delta$. The electricity is free to flow through any number of iterations forwards and backwards through both $\overline\pi$ and $\overline\delta$. Applying a current at some input wire will propogate in such a way that it reaches possible letters which can be reached by some sequence of applications of $\overline\pi$, $\overline\delta$, and their inverses. Then on cable \texttt{A}, applying a current at some input $x$ can only reach wire $y$ on that cable if $\exists\text{ }\sigma\in\langle\overline\pi, \overline\delta\rangle$ such that $\sigma(x) = y$. If we want to know for a given input wire, what other wires it can reach, we need to know which letters are connected to which through permutations in $\langle\overline\pi, \overline\delta\rangle$.
\\\\We note that $\langle\overline\pi, \overline\delta\rangle$ has a natural group action on $\mathbb{N}_{4}$ given by
\begin{align*}
	\sigma\cdot{x} \coloneq \sigma(x)\text{ for }\sigma\in\langle\overline\pi, \overline\delta\rangle\text{ and }x\in\mathbb{N}_{4}.
\end{align*}
Consider an orbit for this group action. That is, for $x\in\mathbb{N}_{4}$ we have
\begin{center}
	$\langle\overline\pi, \overline\delta\rangle\cdot x\coloneq\{\sigma(x)\text{ }|\text{ }\sigma\in\langle\overline\pi, \overline\delta\rangle\}$
\end{center}
\noindent We can think of this orbit as all elements in $\mathbb{N}_{4}$ which $x$ can reach via some sequence finite compositions of $\overline\pi$, $\overline\delta$, and their inverses. This is exactly analogous to the set of wires on cable \texttt{A} which can be reached from an input on $x$. Then these orbits partition $\mathbb{N}_{4}$ and tell us which wires are in a connected loop in our Bombe arrangement.
\\\\In this way the parition given by the set of orbits $\mathbb{N}_{4}/\langle\overline\pi, \overline\delta\rangle$ represents which wires are connected to which wires in our diagram. If $\mathbb{N}_{4}/\langle\overline\pi, \overline\delta\rangle = \{\{\texttt{a},\dots,\texttt{d}\}\}$ then we know that electrifying any wire on cable \texttt{A} will electrify all wires in the diagram. In such a case we say that $\langle\overline\pi, \overline\delta\rangle$ forms a {\bf{transitive subgroup}}.
\\\\Consider how this relates to our single loop example.
\begin{center}
	\begin{tikzpicture}[thick, scale=0.4, every node/.style={scale=0.7}]
		% Draw the box
		\draw[fill=lightgray] (2,-1.5) rectangle (4,2.5) node[midway] {};

		\node at (3, -2) {$\overline\pi_3$};

		% Draw the wires entering the box
		\draw[-] (0, 2) -- (2, 2) node[midway, above] {\texttt{a}};
		\draw[-] (0, 1) -- (2, 1) node[midway, above] {\texttt{b}};
		\draw[-] (0, 0) -- (2, 0) node[midway, above] {\texttt{c}};
		\draw[-] (0,-1) -- (2,-1) node[midway, above] {\texttt{d}};

		% Draw the wires exiting the box with crossed mappings
		\draw[-] (4, 2) -- (6,2) node[midway, above] {\texttt{a}};
		\draw[-] (4, 1) -- (6, 1) node[midway, above] {\texttt{b}};
		\draw[-] (4, 0) -- (6, 0) node[midway, above] {\texttt{c}};
		\draw[-] (4,-1) -- (6, -1) node[midway, above] {\texttt{d}};

		% Draw the lines inside the box to represent the mapping
		\draw[-] (2, 2) -- (4,-1);
		\draw[-] (2, 1) -- (4, 0);
		\draw[-] (2, 0) -- (4, 1);
		\draw[-] (2,-1) -- (4, 2);

		\draw[-] (0-3, 2-7) to[out=180, in=180] (0, 2) node[midway, above] {};
		\draw[-] (0-3, 1-7) to[out=180, in=180] (0, 1) node[midway, above] {};
		\draw[-] (0-3, 0-7) to[out=180, in=180] (0, 0) node[midway, above] {};
		\draw[-] (0-3, -1-7) to[out=180, in=180] (0, -1)
		node[midway, above] {};

		\draw[-] (6+3, 2-7) to[out=360, in=360] (6, 2) node[midway, above] {};
		\draw[-] (6+3, 1-7) to[out=360, in=360] (6, 1) node[midway, above] {};
		\draw[-] (6+3, 0-7) to[out=360, in=360] (6, 0) node[midway, above] {};
		\draw[-] (6+3, -1-7) to[out=360, in=360] (6, -1)
		node[midway, above] {};

		\draw[fill=lightgray] (2-3,-1.5-7) rectangle (4-3,2.5-7) node[midway] {};

		\node at (3-3, -2-7) {$\overline\pi_1$};

		% Draw the wires entering the box
		\draw[-] (0-3, 2-7) -- (2-3, 2-7) node[midway, above] {\texttt{a}};
		\draw[-] (0-3, 1-7) -- (2-3, 1-7) node[midway, above] {\texttt{b}};
		\draw[-] (0-3, 0-7) -- (2-3, 0-7) node[midway, above] {\texttt{c}};
		\draw[-] (0-3,-1-7) -- (2-3,-1-7) node[midway, above] {\texttt{d}};

		% Draw the wires exiting the box
		\draw[-] (4-3, 2-7) -- (6-3,2-7) node[right, above] {\texttt{a}};
		\draw[-] (4-3, 1-7) -- (6-3, 1-7) node[right, above] {\texttt{b}};
		\draw[-] (4-3, 0-7) -- (6-3, 0-7) node[right, above] {\texttt{c}};
		\draw[-] (4-3,-1-7) -- (6-3, -1-7) node[right, above] {\texttt{d}};

		% Draw the lines inside the box to represent the mapping
		\draw[-] (2-3, 2-7) -- (4-3, 0-7);
		\draw[-] (2-3, 1-7) -- (4-3, -1-7);
		\draw[-] (2-3, 0-7) -- (4-3, 2-7);
		\draw[-] (2-3,-1-7) -- (4-3, 1-7);

		\draw[fill=lightgray] (2+3,-1.5-7) rectangle (4+3,2.5-7) node[midway] {};

		\node at (3+3, -2-7) {$\overline\pi_2$};

		% Draw the wires entering the box
		\draw[-] (0+3, 2-7) -- (2+3, 2-7) node[midway, above] {};
		\draw[-] (0+3, 1-7) -- (2+3, 1-7) node[midway, above] {};
		\draw[-] (0+3, 0-7) -- (2+3, 0-7) node[midway, above] {};
		\draw[-] (0+3,-1-7) -- (2+3,-1-7) node[midway, above] {};

		% Draw the wires exiting the box
		\draw[-] (4+3, 2-7) -- (6+3,2-7) node[midway, above] {\texttt{a}};
		\draw[-] (4+3, 1-7) -- (6+3, 1-7) node[midway, above] {\texttt{b}};
		\draw[-] (4+3, 0-7) -- (6+3, 0-7) node[midway, above] {\texttt{c}};
		\draw[-] (4+3,-1-7) -- (6+3, -1-7) node[midway, above] {\texttt{d}};

		\draw[-] (2+3, 2-7) -- (4+3, 0-7);
		\draw[-] (2+3, 1-7) -- (4+3, -1-7);
		\draw[-] (2+3, 0-7) -- (4+3, 2-7);
		\draw[-] (2+3,-1-7) -- (4+3, 1-7);
		\node[draw,circle] at (-4.5, -1) {\texttt{A}};
		\node[draw,circle] at (3, -9) {\texttt{B}};
		\node[draw,circle] at (10.5, -1) {\texttt{C}};
	\end{tikzpicture}
\end{center}
\noindent In this loop, we have that on cable \texttt{A}, wires \texttt{a} and \texttt{d} are connected, and wires \texttt{b} and \texttt{c} are connected. We explained that his could be simply determined by examining the cycle decomposition of $\overline\pi$ which in this case is
\begin{center}
	(\texttt{ad})(\texttt{bc})
\end{center}
\noindent However, we can get the exact same picture by simply considering the set orbits of $\langle\overline\pi\rangle$ acting on $\{\texttt{a},\dots\texttt{z}\}$, which would give us
\begin{center}
	$\{\{\texttt{a}, \texttt{d}\},\{\texttt{b}, \texttt{c}\}\}$
\end{center}
\noindent which gives us an equal picture of the connections between wires on the \texttt{A} cable. Additionally, shifting to this view means that rather than saying the entire loop becomes electrified when $\overline\pi$ has a $26^1$ cycle type, we say that this occurs when $\langle\overline\pi\rangle$ forms a transitive subgroup of $S_4$.
\\\\This description is more robust in that it can handle any number of closures and gives us a picture of which wires are connected to which on a particular cable.
\subsubsection{Distribution of Stops}
Equipped with this mathematical framework are now ready to address Turing's model. We will classify stops by the cardinality and multiplicty of sets in the partition given by the orbits described above. For example, if a set of orbits is
\begin{center}
	$\{\{\texttt{a}, \dots, \texttt{m}\}, \{\texttt{n}, \dots, \texttt{z}\}\}$
\end{center}
we would say this stop is a $13^2$ stop. This is effectively analogous to the cycle type of a permutation but in order to extend this to multiple loops we must frame it in this description of partitions given by orbits of subgroups. We will call these orbit orbit structures.
\\\\In Turing's model, such a $13^2$ stop as above, would count as $0$ normal stops over all steckering hypotheses, In the actual running of the Bombe this would constitute a single stop. Further, a $1^323^1$ stop in Turing's model would be considered as $3$ normal stops over all possible steckering hypotheses. In reality this would only constitute a single real stop in the running of the Bombe. This highlights the two qualms that we are trying to reconcile in Turing's model, the absence of abnormal stops, and the overcounting of normal stops.
\\\\We begin by considering the case of $2$ closures, where at each rotor position of the Bombe, we have some permutations $\overline\pi$ and $\overline\delta$. What is the distribution of stop types? For now we will assume that any loop in a particular configuration of the Bombe represents a random permutation. We will later refine this, but as a heuristic it serves to justify Turing's model. We want to know what stop types are most common. That is, given two random permutations $\overline\pi$ and $\overline\delta$, what is the distribution of orbit structures for the subgroup $\langle\overline\pi, \overline\delta\rangle$? This question was answered by John Dixon in 1969 in his paper ``The Probability of Generating the Symmetric Group.

\begin{theorem}[Dixon's Theorem]
	Given two random permutations $\sigma$ and $\tau$ in $S_n$, the probability that $\langle\sigma, \tau\rangle$ have an orbit structure of $1^{m_1}\dots n^{m_n}$ is
	\begin{align*}
		1-\frac{1}{n!}\sum_{1^{\ell_1}\dots n^{\ell_n} \ne 1^{m_1}\dots n^{m_n}}\prod_{i=1}^n{\frac{(i!t_{i})^{\ell_i}}{\ell_i!}}
	\end{align*}
	where $t_i$ is the probability that two random permutations in $S_i$ form a transitive subgroup.
\end{theorem}
\begin{proof}
	Let $t_i$ be the probability that for two random $x,y\in S_i$, we have that $\langle x,y\rangle$ acts transitively on $\mathbb{N}_i$.
	\\\\Fix a partition $\omega_1\cup\dots\cup\omega_k$ of $\mathbb{N}_n$. We want to know the number of $x,y\in S_n$ such that $\langle x,y\rangle$ have orbits which are exactly the sets in our partition. For $\langle x, y\rangle$ to contain an orbit $\omega_i$, we must have that $\langle x,y\rangle$ acts transitively on $\omega_i$. That is, $\forall$ $a,b\in\omega_i$\text{ }$\exists\text{ }\sigma\in\langle x,y\rangle$ such that $\sigma(a) = b$. This also means that the restriction of the action $\langle x,y\rangle$ to $\omega_i$ is a transitive group action.  The total number of restrictions to $\omega_i$ without any constraints on transitivity is $(|\omega_i|!)^2$, so to get the number of restrictions which represent a transitive action we compute $(|\omega_i|!)^2t_{|\omega_i|}$. To contain all orbits $\omega_i$, we must have that when restricted to each $\omega_i$, $\langle x,y\rangle$ acts transitively. Thus there are
	\begin{center}
		$\prod_{i=1}^k(|\omega_i|!)^2t_{|\omega_i|}$
	\end{center}
	permutations $x,y\in S_n$ such that $\langle x,y\rangle$ has an orbit structure equivalent to the partition $\omega_1\cup\dots\cup\omega_k$.
	\\\\Supposing we have a particular partition with $ell_i$ sets of size $i$ then our above argument can be framed as stating that there are
	\begin{center}
		$\prod_{i=1}^n{((i!)^2t_i)^{\ell_i}}$
	\end{center}
	The number of partitions with $\ell_i$ sets of size $i$ is exactly
	\begin{center}
		$\frac{n!}{\prod_{i=1}^n{(i!)^{\ell_i}\ell_i!}}$
	\end{center}
	We compute the probability that $x,y\in S_n$ do \emph{not} generate an orbit structure with $m_i$ orbits of size $i$, by computing the total number of $x,y\in S_n$ which generate any other orbit structure, and dividing by $(n!)^2$ (i.e. the total number of pairs of permutations in $S_n$). This is precisely
	\begin{align*}
		          & \sum_{1^{\ell_1}\dots n^{\ell_n} \ne 1^{m_1}\dots n^{m_n}}{\frac{1}{(n!)^2}\frac{n!}{\prod_{i=1}^n{(i!)^{\ell_i}\ell_i!}}}\prod_{i=1}^n{((i!)^2t_i)^{\ell_i}} \\
		=\text{ } & \frac{1}{n!}\sum_{1^{\ell_1}\dots n^{\ell_n} \ne 1^{m_1}\dots n^{m_n}}{\prod_{i=1}^n{\frac{(i!)^{2\ell_i}t_i^{\ell_i}}{(i!)^{\ell_i}\ell_i!}}}                \\
		=\text{ } & \frac{1}{n!}\sum_{1^{\ell_1}\dots n^{\ell_n} \ne 1^{m_1}\dots n^{m_n}}{\prod_{i=1}^n{\frac{(i!)^{\ell_i}t_i^{\ell_i}}{
		\ell_i!}}}
	\end{align*}
	The opposite of this probability is the probability that we generate exactly the orbit structure we specified, thus giving us the desired result.
\end{proof}
We can now compute for the case of two loops, what the probability of a particular stop configuration is. Computing the above for a $26^1$ orbit structure tells us that roughly $95.9\%$ of all rotor configurations in a two loop structure will electrify all wires, producing no stop whatsoever. The remaining $4.1\%$ of rotor configurations account for all possible stops. Of these stops we can compute that roughly $92.3\%$ of these stops have a stop type of $1^125^1$. This is a \emph{massive} proportion of the possible stops. Nearly all stops have this singular stop type, and herein lies the justification for Turing's method.
%\\\\With two loops if we could just approximate the number of $1^125^1$ stops we would have a reasonably close approximation for all stops. 
% \\\\For now we will assume that any loop in a particular configuration of the Bombe represents a random permutation. We will later refine this, but as a heuristic it serves to justify Turing's model.
% \\\\For a single loop this boils down to the question of what the most common cycle types are.
% We will first consider the case of an Enigma machine with no diagonal connections and only a single loop. Recall from section \ref{stop} that, in this case, a stop occurs whenever the machine is in such an arrangement that $\overline{\pi}$ (i.e. the permutation representing a loop in our Bombe) is not a $26^1$ cycle. There are many ways in which this can occur. In particular, every cycle type corresponds to a partition of $26$ of which there are $2436$, so there are $2435$ total cycle types which can produce a stop.
% \\\\To develop a comprehensive model which correlated menus to their expected number of stops, we would need to compute, for a given menu, the likelihood of each of these $2435$ cycle types occuring. Given the vast amount of arrangments of cycle types and menu configurations, Turing needed to make some simplifying assumptions.
% \\\\First, we will restrict ourselves to only considering those stops which Turing called {\bf{normal stops}}. He explained that normal stops are ``positions at which by altering the point at which the current enters the diagonal board, one can make 25 relays close.'' In the case of a single loop in our menu, this is equivalent to the statement that the resulting loop in the Bombe has a singleton cycle. If we apply current to this singleton cycle all the remaining relays will close.
% \\\\Turing further restricted the space of menus by requiring that we only examine menus forming a singular connected component.
% % \begin{itemize}
% % 	\item The model will only consider the expected number of stops with cycle type. Turing called these {\bf{normal stops}}. These stops appeared to be the most common form encountered when actually running the Bombe so we can make a heuristic argument that computing the expected number of normal stops will be a reasonable approximation of the actual number of stops.
% % 	\item The model will only consider menus which form a single connected component.
% % \end{itemize}
% \\\\Turing considered a menu in which no loops occured which he called a {\bf{web}}. In this case every position of the Bombe would create a normal stop as there is no feedback necessary to electrify any additional wires on a given cable. Turing considers not only the $26^3$ possible rotor positions of the Bombe, but also the $26$ initial steckering hypotheses we could input. Some steckering hypotheses may produce a normal stop while others may not. In our case, all steckering hypotheses and all rotor configurations produce a normal stop so we get $26^4$ total normal stops. In our simplified model with only $4$ characters this may look as follows
% \subsection{Issues with Turing's Model}
% There are a number of assumptions in Turing's model that may seem reasonable from a heuristic point of view but turn out to have an impact significant enough to deviate the model from simulated ground truth.

% \subsubsection{Normal Stops}
% We begin with the strongest assumption which is that the restriction to the expected number of {\emph{normal}} stops should produce an approximation of the expected number of \emph{all} stops. Turing's model claims that over all stecker hypotheses and rotor positions, a menu with one loop should produce $26^3$ total normal stops, which is implied to give a rough estimate for the number of all stops.
% \\\\In the actual Bombe, a stop occurs whenever $\overline{\pi}$ is not a $26^1$ cycle. To consider the accuracy of Turing's estimate, consider a menu with $4$ characters arranged in a loop. This loop is made of $4$ Enigma permutations $\overline{\pi_1},\dots, \overline{\pi_4}$. Each $\overline\pi_i$ has a $2^{13}$ cycle type and thus has odd parity. Composing four such permutations to get our loop permutation $\overline\pi$ we know that this resulting permutation must thus have an even parity. However, a $26^1$ cycle has odd parity. It then follows that the composition of four Enigma permutations can \emph{never} result in a $26^1$ cycle. Thus such an arrangment of scramblers will \emph{always} stop. Thus, for such a menu, all $26^3$ rotor positions of the Bombe will cause a stop. Given that the stopping criterion is independent of choice of stecker hypothesis, to align with Turing's model we can simply multiply by all $26$ initial steckering hypotheses to deduce that for such a menu we will get $26^4$ total stops. This is a significant disagreement with Turing's equation.
% \\\\Perhaps it would align closer if we had an odd number of letters in our menu. Simulating a menu with $5$ letters
%\\\\Second the
% In this chapter we will derive the expected number of stops for
% particular arrangments of the machine. The machine is wired to stop
% when $\overline\sigma$ has cycle type other than $(26)$. Turing only
% considers what he calls \textbf{normal stops} during his calculation
% of the expected number of stops. This is a stop which has cycle-type $(25,1)$.
% We will attempt to expand to a consideration of all possible stops.
% \subsection{Prior Work}
% \subsection{Stops Without Diagonal Board}
% \begin{center}
%   \begin{tikzpicture}[thick, scale=0.6, every node/.style={scale=0.7}]
%     % Draw the box
%     \draw[fill=lightgray] (2,-1.5) rectangle (4,2.5) node[midway] {};

%     \node at (3, -2) {$\overline\sigma_3$};

%     % Draw the wires entering the box
%     \draw[-] (0, 2) -- (2, 2) node[midway, above] {a};
%     \draw[-] (0, 1) -- (2, 1) node[midway, above] {b};
%     \draw[-] (0, 0) -- (2, 0) node[midway, above] {c};
%     \draw[-] (0,-1) -- (2,-1) node[midway, above] {d};

%     % Draw the wires exiting the box with crossed mappings
%     \draw[-] (4, 2) -- (6,2) node[midway, above] {a};
%     \draw[-] (4, 1) -- (6, 1) node[midway, above] {b};
%     \draw[-] (4, 0) -- (6, 0) node[midway, above] {c};
%     \draw[-] (4,-1) -- (6, -1) node[midway, above] {d};

%     % Draw the lines inside the box to represent the mapping
%     \draw[-] (2, 2) -- (4,-1);
%     \draw[-] (2, 1) -- (4, 0);
%     \draw[-] (2, 0) -- (4, 1);
%     \draw[-] (2,-1) -- (4, 2);

%     \draw[-] (0-3, 2-7) to[out=180, in=180] (0, 2) node[midway, above] {};
%     \draw[-] (0-3, 1-7) to[out=180, in=180] (0, 1) node[midway, above] {};
%     \draw[-] (0-3, 0-7) to[out=180, in=180] (0, 0) node[midway, above] {};
%     \draw[-] (0-3, -1-7) to[out=180, in=180] (0, -1) node[midway, above] {};

%     \draw[-] (6+3, 2-7) to[out=360, in=360] (6, 2) node[midway, above] {};
%     \draw[-] (6+3, 1-7) to[out=360, in=360] (6, 1) node[midway, above] {};
%     \draw[-] (6+3, 0-7) to[out=360, in=360] (6, 0) node[midway, above] {};
%     \draw[-] (6+3, -1-7) to[out=360, in=360] (6, -1) node[midway, above] {};

%     \draw[fill=lightgray] (2-3,-1.5-7) rectangle (4-3,2.5-7) node[midway] {};

%     \node at (3-3, -2-7) {$\overline\sigma_1$};

%     % Draw the wires entering the box
%     \draw[-] (0-3, 2-7) -- (2-3, 2-7) node[midway, above] {a};
%     \draw[-] (0-3, 1-7) -- (2-3, 1-7) node[midway, above] {b};
%     \draw[-] (0-3, 0-7) -- (2-3, 0-7) node[midway, above] {c};
%     \draw[-] (0-3,-1-7) -- (2-3,-1-7) node[midway, above] {d};

%     % Draw the wires exiting the box
%     \draw[-] (4-3, 2-7) -- (6-3,2-7) node[right, above] {a};
%     \draw[-] (4-3, 1-7) -- (6-3, 1-7) node[right, above] {b};
%     \draw[-] (4-3, 0-7) -- (6-3, 0-7) node[right, above] {c};
%     \draw[-] (4-3,-1-7) -- (6-3, -1-7) node[right, above] {d};

%     % Draw the lines inside the box to represent the mapping
%     \draw[-] (2-3, 2-7) -- (4-3, 0-7);
%     \draw[-] (2-3, 1-7) -- (4-3, -1-7);
%     \draw[-] (2-3, 0-7) -- (4-3, 2-7);
%     \draw[-] (2-3,-1-7) -- (4-3, 1-7);

%     \draw[fill=lightgray] (2+3,-1.5-7) rectangle (4+3,2.5-7) node[midway] {};

%     \node at (3+3, -2-7) {$\overline\sigma_2$};

%     % Draw the wires entering the box
%     \draw[-] (0+3, 2-7) -- (2+3, 2-7) node[midway, above] {};
%     \draw[-] (0+3, 1-7) -- (2+3, 1-7) node[midway, above] {};
%     \draw[-] (0+3, 0-7) -- (2+3, 0-7) node[midway, above] {};
%     \draw[-] (0+3,-1-7) -- (2+3,-1-7) node[midway, above] {};

%     % Draw the wires exiting the box
%     \draw[-] (4+3, 2-7) -- (6+3,2-7) node[midway, above] {a};
%     \draw[-] (4+3, 1-7) -- (6+3, 1-7) node[midway, above] {b};
%     \draw[-] (4+3, 0-7) -- (6+3, 0-7) node[midway, above] {c};
%     \draw[-] (4+3,-1-7) -- (6+3, -1-7) node[midway, above] {d};

%     \draw[-] (2+3, 2-7) -- (4+3, 1-7);
%     \draw[-] (2+3, 1-7) -- (4+3, 2-7);
%     \draw[-] (2+3, 0-7) -- (4+3, -1-7);
%     \draw[-] (2+3,-1-7) -- (4+3, 0-7);

%     \node[draw,circle] at (-4.5, -1) {$A$};
%     \node[draw,circle] at (3, -9) {$B$};
%     \node[draw,circle] at (10.5, -1) {$C$};

%   \end{tikzpicture}
% \end{center}
% Consider our simple example of a loop of three Enigmas on four
% letters. We might expect that $\overline\sigma =
%   \overline\sigma_3\overline\sigma_2\overline\sigma_1$ being generated
% from considerably random permutations, is itself a random
% permutation. If this is the case then we would expect that we would
% get a $(4)$ cycle with a probability of $\frac{1}{4}$. Then we expect
% the machine to stop
% with probability $\frac{3}{4}$. With enough loops this probability
% decreases exponentially and the machine has a tractible number of stops.
% However, try as we may, we can never find a collection of Enigma
% permutations $\{\overline\sigma_1, \overline\sigma_2,
%   \overline\sigma_3\}$ which generate a $(4)$ cycle in
% $\overline\sigma$. This is to say, in our above arrangment, the
% machine will stop at \emph{every} rotor position thus making the
% process of checking stops intractible.
% \\\\To see why this is the case, note that each $\overline\sigma_i$
% has cycle type $(2,2)$ thus they are permutations of even parity. On
% the other hand, any $(4)$ cycle will have odd parity.
% When we compose $3$ even permutations (i.e.
% $\overline\sigma_3\overline\sigma_2\overline\sigma_1$) we will always
% get an even parity permutation, thus this resulting permutation can
% \emph{never} be a $(4)$ cycle.
% \\\\In the case of the Bombe, a cycle of even length can never
% produce a permutation with a $(26)$ cycle. We can emperically observe
% this by simulating the Bombe's operation on a cycle of length $8$ and
% we find that every single rotor
% position produces a stop.
% \\\\From the above it is clear that $\overline\sigma$ is certainly
% not a purely random permutation, and simulations of loops of Enigma
% permutations of various lengths show that the probability
% distribution of these permutations is highly dependent on
% the length of the loop. A table for estimated probabilities for
% Enigma machines on $4$ letters is shown below.
% \\\\NOTE THAT WE ARE ASSUMING AN ENIGMA MACHINE IS A RANDOM 2,2... CYCLE
% \\\\Given how skewed the distribution of permutations are for each
% cycle length it follows naturally that to express the probability of
% a stop with a particular machine arrangement should not just be a function
% of the number of closures or letters in a menu, but rather the
% particular length of each closure.
% \subsubsection{Singular Loop}\text{}
% \\\\For the case of a single loop we can run a simulation to estimate
% the probability of a stop given the length of a loop. This is akin to
% the table above though we combine the total probabilities of any
% cycle type other than $(26)$ to get a probability of a stop
% for each length of loop.
% \subsubsection{Multiple Loops}\text{}
% \\\\Multiple loops presents some complexities. Initally one might
% suspect that these loops are independent, and while the cycle type
% probabilities of each loop may be independent, due to the electrical
% interconnections between the loops we need to take more into account.
% For example, we have noted that an even loop length of Enigma
% machines will always stop. However, two loops of odd length may
% result in a configuration that will not cause a stop. To see this
% consider our example on $4$ letters. In this case, an odd length loop
% can never generate a $(4)$ cycle.
% Consider now the following loops of Enigma permutations representing
% permutations $\overline\sigma$ and $\overline\delta$ respectively.
% \subsection{Introducing the Diagonal Board}
% \begin{center}
%   \begin{tikzpicture}[thick, scale=0.6, every node/.style={scale=0.7}]
%     % Draw the box
%     \draw[fill=lightgray] (2,-1.5) rectangle (4,2.5) node[midway] {};

%     \node at (3, -2) {$\overline\sigma_3$};

%     % Draw the wires entering the box
%     \draw[-] (0, 2) -- (2, 2) node[midway, above] {a};
%     \draw[-] (0, 1) -- (2, 1) node[midway, above] {b};
%     \draw[-] (0, 0) -- (2, 0) node[midway, above] {c};
%     \draw[-] (0,-1) -- (2,-1) node[midway, above] {d};

%     % Draw the wires exiting the box with crossed mappings
%     \draw[-] (4, 2) -- (6,2) node[midway, above] {a};
%     \draw[-] (4, 1) -- (6, 1) node[midway, above] {b};
%     \draw[-] (4, 0) -- (6, 0) node[midway, above] {c};
%     \draw[-] (4,-1) -- (6, -1) node[midway, above] {d};

%     % Draw the lines inside the box to represent the mapping
%     \draw[-] (2, 2) -- (4,-1);
%     \draw[-] (2, 1) -- (4, 0);
%     \draw[-] (2, 0) -- (4, 1);
%     \draw[-] (2,-1) -- (4, 2);

%     \draw[-] (0-3, 2-7) to[out=180, in=180] (0, 2) node[midway, above] {};
%     \draw[-] (0-3, 1-7) to[out=180, in=180] (0, 1) node[midway, above] {};
%     \draw[-] (0-3, 0-7) to[out=180, in=180] (0, 0) node[midway, above] {};
%     \draw[-] (0-3, -1-7) to[out=180, in=180] (0, -1) node[midway, above] {};

%     \draw[-] (6+3, 2-7) to[out=360, in=360] (6, 2) node[midway, above] {};
%     \draw[-] (6+3, 1-7) to[out=360, in=360] (6, 1) node[midway, above] {};
%     \draw[-] (6+3, 0-7) to[out=360, in=360] (6, 0) node[midway, above] {};
%     \draw[-] (6+3, -1-7) to[out=360, in=360] (6, -1) node[midway, above] {};

%     \draw[fill=lightgray] (2-3,-1.5-7) rectangle (4-3,2.5-7) node[midway] {};

%     \node at (3-3, -2-7) {$\overline\sigma_1$};

%     % Draw the wires entering the box
%     \draw[-] (0-3, 2-7) -- (2-3, 2-7) node[midway, above] {a};
%     \draw[-] (0-3, 1-7) -- (2-3, 1-7) node[midway, above] {b};
%     \draw[-] (0-3, 0-7) -- (2-3, 0-7) node[midway, above] {c};
%     \draw[-] (0-3,-1-7) -- (2-3,-1-7) node[midway, above] {d};

%     % Draw the wires exiting the box
%     \draw[-] (4-3, 2-7) -- (6-3,2-7) node[right, above] {a};
%     \draw[-] (4-3, 1-7) -- (6-3, 1-7) node[right, above] {b};
%     \draw[-] (4-3, 0-7) -- (6-3, 0-7) node[right, above] {c};
%     \draw[-] (4-3,-1-7) -- (6-3, -1-7) node[right, above] {d};

%     % Draw the lines inside the box to represent the mapping
%     \draw[-] (2-3, 2-7) -- (4-3, 0-7);
%     \draw[-] (2-3, 1-7) -- (4-3, -1-7);
%     \draw[-] (2-3, 0-7) -- (4-3, 2-7);
%     \draw[-] (2-3,-1-7) -- (4-3, 1-7);

%     \draw[fill=lightgray] (2+3,-1.5-7) rectangle (4+3,2.5-7) node[midway] {};

%     \node at (3+3, -2-7) {$\overline\sigma_2$};

%     % Draw the wires entering the box
%     \draw[-] (0+3, 2-7) -- (2+3, 2-7) node[midway, above] {};
%     \draw[-] (0+3, 1-7) -- (2+3, 1-7) node[midway, above] {};
%     \draw[-] (0+3, 0-7) -- (2+3, 0-7) node[midway, above] {};
%     \draw[-] (0+3,-1-7) -- (2+3,-1-7) node[midway, above] {};

%     % Draw the wires exiting the box
%     \draw[-] (4+3, 2-7) -- (6+3,2-7) node[midway, above] {a};
%     \draw[-] (4+3, 1-7) -- (6+3, 1-7) node[midway, above] {b};
%     \draw[-] (4+3, 0-7) -- (6+3, 0-7) node[midway, above] {c};
%     \draw[-] (4+3,-1-7) -- (6+3, -1-7) node[midway, above] {d};

%     \draw[-] (2+3, 2-7) -- (4+3, 1-7);
%     \draw[-] (2+3, 1-7) -- (4+3, 2-7);
%     \draw[-] (2+3, 0-7) -- (4+3, -1-7);
%     \draw[-] (2+3,-1-7) -- (4+3, 0-7);

%     \node[draw,circle] at (-4.5, -1) {$A$};
%     \node[draw,circle] at (3, -9) {$B$};
%     \node[draw,circle] at (10.5, -1) {$C$};

%   \end{tikzpicture}
%   \begin{tikzpicture}[thick, scale=0.6, every node/.style={scale=0.7}]
%     % Draw the box
%     \draw[fill=lightgray] (2,-1.5) rectangle (4,2.5) node[midway] {};

%     \node at (3, -2) {$\overline\delta_3$};

%     % Draw the wires entering the box
%     \draw[-] (0, 2) -- (2, 2) node[midway, above] {a};
%     \draw[-] (0, 1) -- (2, 1) node[midway, above] {b};
%     \draw[-] (0, 0) -- (2, 0) node[midway, above] {c};
%     \draw[-] (0,-1) -- (2,-1) node[midway, above] {d};

%     % Draw the wires exiting the box with crossed mappings
%     \draw[-] (4, 2) -- (6,2) node[midway, above] {a};
%     \draw[-] (4, 1) -- (6, 1) node[midway, above] {b};
%     \draw[-] (4, 0) -- (6, 0) node[midway, above] {c};
%     \draw[-] (4,-1) -- (6, -1) node[midway, above] {d};

%     % Draw the lines inside the box to represent the mapping
%     \draw[-] (2, 2) -- (4,-1);
%     \draw[-] (2, 1) -- (4, 0);
%     \draw[-] (2, 0) -- (4, 1);
%     \draw[-] (2,-1) -- (4, 2);

%     \draw[-] (0-3, 2-7) to[out=180, in=180] (0, 2) node[midway, above] {};
%     \draw[-] (0-3, 1-7) to[out=180, in=180] (0, 1) node[midway, above] {};
%     \draw[-] (0-3, 0-7) to[out=180, in=180] (0, 0) node[midway, above] {};
%     \draw[-] (0-3, -1-7) to[out=180, in=180] (0, -1) node[midway, above] {};

%     \draw[-] (6+3, 2-7) to[out=360, in=360] (6, 2) node[midway, above] {};
%     \draw[-] (6+3, 1-7) to[out=360, in=360] (6, 1) node[midway, above] {};
%     \draw[-] (6+3, 0-7) to[out=360, in=360] (6, 0) node[midway, above] {};
%     \draw[-] (6+3, -1-7) to[out=360, in=360] (6, -1) node[midway, above] {};

%     \draw[fill=lightgray] (2-3,-1.5-7) rectangle (4-3,2.5-7) node[midway] {};

%     \node at (3-3, -2-7) {$\overline\delta_1$};

%     % Draw the wires entering the box
%     \draw[-] (0-3, 2-7) -- (2-3, 2-7) node[midway, above] {a};
%     \draw[-] (0-3, 1-7) -- (2-3, 1-7) node[midway, above] {b};
%     \draw[-] (0-3, 0-7) -- (2-3, 0-7) node[midway, above] {c};
%     \draw[-] (0-3,-1-7) -- (2-3,-1-7) node[midway, above] {d};

%     % Draw the wires exiting the box
%     \draw[-] (4-3, 2-7) -- (6-3,2-7) node[right, above] {a};
%     \draw[-] (4-3, 1-7) -- (6-3, 1-7) node[right, above] {b};
%     \draw[-] (4-3, 0-7) -- (6-3, 0-7) node[right, above] {c};
%     \draw[-] (4-3,-1-7) -- (6-3, -1-7) node[right, above] {d};

%     % Draw the lines inside the box to represent the mapping
%     \draw[-] (2-3, 2-7) -- (4-3, 0-7);
%     \draw[-] (2-3, 1-7) -- (4-3, -1-7);
%     \draw[-] (2-3, 0-7) -- (4-3, 2-7);
%     \draw[-] (2-3,-1-7) -- (4-3, 1-7);

%     \draw[fill=lightgray] (2+3,-1.5-7) rectangle (4+3,2.5-7) node[midway] {};

%     \node at (3+3, -2-7) {$\overline\delta_2$};

%     % Draw the wires entering the box
%     \draw[-] (0+3, 2-7) -- (2+3, 2-7) node[midway, above] {};
%     \draw[-] (0+3, 1-7) -- (2+3, 1-7) node[midway, above] {};
%     \draw[-] (0+3, 0-7) -- (2+3, 0-7) node[midway, above] {};
%     \draw[-] (0+3,-1-7) -- (2+3,-1-7) node[midway, above] {};

%     % Draw the wires exiting the box
%     \draw[-] (4+3, 2-7) -- (6+3,2-7) node[midway, above] {a};
%     \draw[-] (4+3, 1-7) -- (6+3, 1-7) node[midway, above] {b};
%     \draw[-] (4+3, 0-7) -- (6+3, 0-7) node[midway, above] {c};
%     \draw[-] (4+3,-1-7) -- (6+3, -1-7) node[midway, above] {d};

%     \draw[-] (2+3, 2-7) -- (4+3, 1-7);
%     \draw[-] (2+3, 1-7) -- (4+3, 2-7);
%     \draw[-] (2+3, 0-7) -- (4+3, -1-7);
%     \draw[-] (2+3,-1-7) -- (4+3, 0-7);

%     \node[draw,circle] at (-4.5, -1) {$A$};
%     \node[draw,circle] at (3, -9) {$B$};
%     \node[draw,circle] at (10.5, -1) {$C$};

%   \end{tikzpicture}
% \end{center}

% \begin{center}
%   \begin{tikzpicture}[thick, scale=0.6, every node/.style={scale=0.7}]
%     % Draw the box
%     \draw[fill=lightgray] (2,-1.5) rectangle (4,2.5) node[midway] {};

%     \node at (3, -2) {$\overline\sigma_3$};

%     % Draw the wires entering the box
%     \draw[-, red] (0, 2) -- (2, 2) node[midway, above] {a};
%     \draw[-] (0, 1) -- (2, 1) node[midway, above] {b};
%     \draw[-, red] (0, 0) -- (2, 0) node[midway, above] {c};
%     \draw[-] (0,-1) -- (2,-1) node[midway, above] {d};

%     % Draw the wires exiting the box with crossed mappings
%     \draw[-, red] (4, 2) -- (6,2) node[midway, above] {a};
%     \draw[-] (4, 1) -- (6, 1) node[midway, above] {b};
%     \draw[-, red] (4, 0) -- (6, 0) node[midway, above] {c};
%     \draw[-] (4,-1) -- (6, -1) node[midway, above] {d};

%     % Draw the lines inside the box to represent the mapping
%     \draw[-, red] (2, 2) -- (4,0);
%     \draw[-] (2, 1) -- (4, -1);
%     \draw[-, red] (2, 0) -- (4, 2);
%     \draw[-] (2,-1) -- (4, 1);

%     \draw[-, red] (0-3, 2-7) to[out=180, in=180] (0, 2)
% node[midway, above] {};
%     \draw[-] (0-3, 1-7) to[out=180, in=180] (0, 1) node[midway, above] {};
%     \draw[-, red] (0-3, 0-7) to[out=180, in=180] (0, 0)
% node[midway, above] {};
%     \draw[-] (0-3, -1-7) to[out=180, in=180] (0, -1) node[midway, above] {};

%     \draw[-, red] (6+3, 2-7) to[out=360, in=360] (6, 2)
% node[midway, above] {};
%     \draw[-] (6+3, 1-7) to[out=360, in=360] (6, 1) node[midway, above] {};
%     \draw[-, red] (6+3, 0-7) to[out=360, in=360] (6, 0)
% node[midway, above] {};
%     \draw[-] (6+3, -1-7) to[out=360, in=360] (6, -1) node[midway, above] {};

%     \draw[fill=lightgray] (2-3,-1.5-7) rectangle (4-3,2.5-7) node[midway] {};

%     \node at (3-3, -2-7) {$\overline\sigma_1$};

%     % Draw the wires entering the box
%     \draw[-, red] (0-3, 2-7) -- (2-3, 2-7) node[midway, above] {a};
%     \draw[-] (0-3, 1-7) -- (2-3, 1-7) node[midway, above] {b};
%     \draw[-, red] (0-3, 0-7) -- (2-3, 0-7) node[midway, above] {c};
%     \draw[-] (0-3,-1-7) -- (2-3,-1-7) node[midway, above] {d};

%     % Draw the wires exiting the box
%     \draw[-, red] (4-3, 2-7) -- (6-3,2-7) node[right, above] {a};
%     \draw[-] (4-3, 1-7) -- (6-3, 1-7) node[right, above] {b};
%     \draw[-, red] (4-3, 0-7) -- (6-3, 0-7) node[right, above] {c};
%     \draw[-] (4-3,-1-7) -- (6-3, -1-7) node[right, above] {d};

%     % Draw the lines inside the box to represent the mapping
%     \draw[-, red] (2-3, 2-7) -- (4-3, 0-7);
%     \draw[-] (2-3, 1-7) -- (4-3, -1-7);
%     \draw[-, red] (2-3, 0-7) -- (4-3, 2-7);
%     \draw[-] (2-3,-1-7) -- (4-3, 1-7);

%     \draw[fill=lightgray] (2+3,-1.5-7) rectangle (4+3,2.5-7) node[midway] {};

%     \node at (3+3, -2-7) {$\overline\sigma_2$};

%     % Draw the wires entering the box
%     \draw[-, red] (0+3, 2-7) -- (2+3, 2-7) node[midway, above] {};
%     \draw[-] (0+3, 1-7) -- (2+3, 1-7) node[midway, above] {};
%     \draw[-, red] (0+3, 0-7) -- (2+3, 0-7) node[midway, above] {};
%     \draw[-] (0+3,-1-7) -- (2+3,-1-7) node[midway, above] {};

%     % Draw the wires exiting the box
%     \draw[-, red] (4+3, 2-7) -- (6+3,2-7) node[midway, above] {a};
%     \draw[-] (4+3, 1-7) -- (6+3, 1-7) node[midway, above] {b};
%     \draw[-, red] (4+3, 0-7) -- (6+3, 0-7) node[midway, above] {c};
%     \draw[-] (4+3,-1-7) -- (6+3, -1-7) node[midway, above] {d};

%     \draw[-, red] (2+3, 2-7) -- (4+3, 0-7);
%     \draw[-] (2+3, 1-7) -- (4+3, -1-7);
%     \draw[-, red] (2+3, 0-7) -- (4+3, 2-7);
%     \draw[-] (2+3,-1-7) -- (4+3, 1-7);

%     \node[draw,circle] at (-4.5, -1) {$A$};
%     \node[draw,circle] at (3, -9) {$B$};
%     \node[draw,circle] at (10.5, -1) {$C$};
%   \end{tikzpicture}
%   \begin{tikzpicture}[thick, scale=0.6, every node/.style={scale=0.7}]
%     % Draw the box
%     \draw[fill=lightgray] (2,-1.5) rectangle (4,2.5) node[midway] {};

%     \node at (3, -2) {$\overline\delta_3$};

%     % Draw the wires entering the box
%     \draw[-, red] (0, 2) -- (2, 2) node[midway, above] {a};
%     \draw[-] (0, 1) -- (2, 1) node[midway, above] {b};
%     \draw[-] (0, 0) -- (2, 0) node[midway, above] {c};
%     \draw[-, red] (0,-1) -- (2,-1) node[midway, above] {d};

%     % Draw the wires exiting the box with crossed mappings
%     \draw[-, red] (4, 2) -- (6,2) node[midway, above] {a};
%     \draw[-] (4, 1) -- (6, 1) node[midway, above] {b};
%     \draw[-] (4, 0) -- (6, 0) node[midway, above] {c};
%     \draw[-, red] (4,-1) -- (6, -1) node[midway, above] {d};

%     % Draw the lines inside the box to represent the mapping
%     \draw[-, red] (2, 2) -- (4,-1);
%     \draw[-] (2, 1) -- (4, 0);
%     \draw[-] (2, 0) -- (4, 1);
%     \draw[-, red] (2,-1) -- (4, 2);

%     \draw[-, red] (0-3, 2-7) to[out=180, in=180] (0, 2)
% node[midway, above] {};
%     \draw[-] (0-3, 1-7) to[out=180, in=180] (0, 1) node[midway, above] {};
%     \draw[-] (0-3, 0-7) to[out=180, in=180] (0, 0) node[midway, above] {};
%     \draw[-, red] (0-3, -1-7) to[out=180, in=180] (0, -1)
%     node[midway, above] {};

%     \draw[-, red] (6+3, 2-7) to[out=360, in=360] (6, 2)
% node[midway, above] {};
%     \draw[-] (6+3, 1-7) to[out=360, in=360] (6, 1) node[midway, above] {};
%     \draw[-] (6+3, 0-7) to[out=360, in=360] (6, 0) node[midway, above] {};
%     \draw[-, red] (6+3, -1-7) to[out=360, in=360] (6, -1)
%     node[midway, above] {};

%     \draw[fill=lightgray] (2-3,-1.5-7) rectangle (4-3,2.5-7) node[midway] {};

%     \node at (3-3, -2-7) {$\overline\delta_1$};

%     % Draw the wires entering the box
%     \draw[-, red] (0-3, 2-7) -- (2-3, 2-7) node[midway, above] {a};
%     \draw[-] (0-3, 1-7) -- (2-3, 1-7) node[midway, above] {b};
%     \draw[-] (0-3, 0-7) -- (2-3, 0-7) node[midway, above] {c};
%     \draw[-, red] (0-3,-1-7) -- (2-3,-1-7) node[midway, above] {d};

%     % Draw the wires exiting the box
%     \draw[-] (4-3, 2-7) -- (6-3,2-7) node[right, above] {a};
%     \draw[-, red] (4-3, 1-7) -- (6-3, 1-7) node[right, above] {b};
%     \draw[-, red] (4-3, 0-7) -- (6-3, 0-7) node[right, above] {c};
%     \draw[-] (4-3,-1-7) -- (6-3, -1-7) node[right, above] {d};

%     % Draw the lines inside the box to represent the mapping
%     \draw[-, red] (2-3, 2-7) -- (4-3, 0-7);
%     \draw[-] (2-3, 1-7) -- (4-3, -1-7);
%     \draw[-] (2-3, 0-7) -- (4-3, 2-7);
%     \draw[-, red] (2-3,-1-7) -- (4-3, 1-7);

%     \draw[fill=lightgray] (2+3,-1.5-7) rectangle (4+3,2.5-7) node[midway] {};

%     \node at (3+3, -2-7) {$\overline\sigma_2$};

%     % Draw the wires entering the box
%     \draw[-] (0+3, 2-7) -- (2+3, 2-7) node[midway, above] {};
%     \draw[-, red] (0+3, 1-7) -- (2+3, 1-7) node[midway, above] {};
%     \draw[-, red] (0+3, 0-7) -- (2+3, 0-7) node[midway, above] {};
%     \draw[-] (0+3,-1-7) -- (2+3,-1-7) node[midway, above] {};

%     % Draw the wires exiting the box
%     \draw[-, red] (4+3, 2-7) -- (6+3,2-7) node[midway, above] {a};
%     \draw[-] (4+3, 1-7) -- (6+3, 1-7) node[midway, above] {b};
%     \draw[-] (4+3, 0-7) -- (6+3, 0-7) node[midway, above] {c};
%     \draw[-, red] (4+3,-1-7) -- (6+3, -1-7) node[midway, above] {d};

%     \draw[-] (2+3, 2-7) -- (4+3, 0-7);
%     \draw[-, red] (2+3, 1-7) -- (4+3, -1-7);
%     \draw[-, red] (2+3, 0-7) -- (4+3, 2-7);
%     \draw[-] (2+3,-1-7) -- (4+3, 1-7);

%     \node[draw,circle] at (-4.5, -1) {$A$};
%     \node[draw,circle] at (3, -9) {$B$};
%     \node[draw,circle] at (10.5, -1) {$C$};
%   \end{tikzpicture}
% \end{center}

% \begin{center}
%   \begin{tikzpicture}[thick, scale=0.6, every node/.style={scale=0.7}]
%     % Draw the box
%     \draw[fill=lightgray] (2-6,-1.5) rectangle (4-6,2.5) node[midway] {};

%     \node at (3-6, -2) {$\overline\sigma_3$};

%     % Draw the wires entering the box
%     \draw[-] (0-6, 2) -- (2-6, 2) node[midway, above] {a};
%     \draw[-] (0-6, 1) -- (2-6, 1) node[midway, above] {b};
%     \draw[-] (0-6, 0) -- (2-6, 0) node[midway, above] {c};
%     \draw[-] (0-6,-1) -- (2-6,-1) node[midway, above] {d};

%     % Draw the wires exiting the box with crossed mappings
%     \draw[-] (4-6, 2) -- (6-6,2) node[right, above] {a};
%     \draw[-] (4-6, 1) -- (6-6, 1) node[right, above] {b};
%     \draw[-] (4-6, 0) -- (6-6, 0) node[right, above] {c};
%     \draw[-] (4-6,-1) -- (6-6, -1) node[right, above] {d};

%     % Draw the lines inside the box to represent the mapping
%     \draw[-] (2-6, 2) -- (4-6,0);
%     \draw[-] (2-6, 1) -- (4-6, -1);
%     \draw[-] (2-6, 0) -- (4-6, 2);
%     \draw[-] (2-6,-1) -- (4-6, 1);

%     \draw[fill=lightgray] (2,-1.5) rectangle (4,2.5) node[midway] {};

%     \node at (3, -2) {$\overline\sigma_1 = \overline\delta_1$};

%     % Draw the wires entering the box
%     \draw[-] (0, 2) -- (2, 2) node[midway, above] {};
%     \draw[-] (0, 1) -- (2, 1) node[midway, above] {};
%     \draw[-] (0, 0) -- (2, 0) node[midway, above] {};
%     \draw[-] (0,-1) -- (2,-1) node[midway, above] {};

%     % Draw the wires exiting the box
%     \draw[-] (4, 2) -- (6,2) node[right, above] {a};
%     \draw[-] (4, 1) -- (6, 1) node[right, above] {b};
%     \draw[-] (4, 0) -- (6, 0) node[right, above] {c};
%     \draw[-] (4,-1) -- (6, -1) node[right, above] {d};

%     % Draw the lines inside the box to represent the mapping
%     \draw[-] (2, 2) -- (4, -1);
%     \draw[-] (2, 1) -- (4, 0);
%     \draw[-] (2, 0) -- (4, 1);
%     \draw[-] (2,-1) -- (4, 2);

%     \draw[fill=lightgray] (2+6,-1.5) rectangle (4+6,2.5) node[midway] {};

%     \node at (3+6, -2) {$\overline\sigma_2$};

%     % Draw the wires entering the box
%     \draw[-] (0+6, 2) -- (2+6, 2) node[midway, above] {};
%     \draw[-] (0+6, 1) -- (2+6, 1) node[midway, above] {};
%     \draw[-] (0+6, 0) -- (2+6, 0) node[midway, above] {};
%     \draw[-] (0+6,-1) -- (2+6,-1) node[midway, above] {};

%     % Draw the wires exiting the box
%     \draw[-] (4+6, 2) -- (6+6,2) node[midway, above] {a};
%     \draw[-] (4+6, 1) -- (6+6, 1) node[midway, above] {b};
%     \draw[-] (4+6, 0) -- (6+6, 0) node[midway, above] {c};
%     \draw[-] (4+6,-1) -- (6+6, -1) node[midway, above] {d};

%     \draw[-] (2+6, 2) -- (4+6, 0);
%     \draw[-] (2+6, 1) -- (4+6, -1);
%     \draw[-] (2+6, 0) -- (4+6, 2);
%     \draw[-] (2+6,-1) -- (4+6, 1);

%     \draw[-] (0-6, 2) to[out=180, in=180] (-6, 2-7) node[midway, above] {};
%     \draw[-] (0-6, 1) to[out=180, in=180] (-6, 1-7) node[midway, above] {};
%     \draw[-] (0-6, 0) to[out=180, in=180] (-6, 0-7) node[midway, above] {};
%     \draw[-] (0-6, -1) to[out=180, in=180] (-6, -1-7) node[midway, above] {};

%     \draw[-] (6+6, 2) to[out=360, in=360] (6+6, 2-7) node[midway, above] {};
%     \draw[-] (6+6, 1) to[out=360, in=360] (6+6, 1-7) node[midway, above] {};
%     \draw[-] (6+6, 0) to[out=360, in=360] (6+6, 0-7) node[midway, above] {};
%     \draw[-] (6+6, -1) to[out=360, in=360] (6+6, -1-7) node[midway, above] {};

%     \draw[-] (-6, 2-7) to (6+6, 2-7) node[midway, above] {};
%     \draw[-] (-6, 1-7) to (6+6, 1-7) node[midway, above] {};
%     \draw[-] (-6, 0-7) to (6+6, 0-7) node[midway, above] {};
%     \draw[-] (-6, -1-7) to (6+6, -1-7) node[midway, above] {};

%     \draw[-] (0-6, 2-14) to[out=180, in=180] (-6, 2-7) node[midway, above] {};
%     \draw[-] (0-6, 1-14) to[out=180, in=180] (-6, 1-7) node[midway, above] {};
%     \draw[-] (0-6, 0-14) to[out=180, in=180] (-6, 0-7) node[midway, above] {};
%     \draw[-] (0-6, -1-14) to[out=180, in=180] (-6, -1-7)
% node[midway, above] {};

%     \draw[-] (6+6, 2-14) to[out=360, in=360] (6+6, 2-7)
% node[midway, above] {};
%     \draw[-] (6+6, 1-14) to[out=360, in=360] (6+6, 1-7)
% node[midway, above] {};
%     \draw[-] (6+6, 0-14) to[out=360, in=360] (6+6, 0-7)
% node[midway, above] {};
%     \draw[-] (6+6, -1-14) to[out=360, in=360] (6+6, -1-7)
%     node[midway, above] {};

%     \draw[fill=lightgray] (2-6,-1.5-14) rectangle (4-6,2.5-14)
% node[midway] {};

%     \node at (3-6, -2) {$\overline\sigma_3$};

%     % Draw the wires entering the box
%     \draw[-] (0-6, 2-14) -- (2-6, 2-14) node[midway, above] {a};
%     \draw[-] (0-6, 1-14) -- (2-6, 1-14) node[midway, above] {b};
%     \draw[-] (0-6, 0-14) -- (2-6, 0-14) node[midway, above] {c};
%     \draw[-] (0-6,-1-14) -- (2-6,-1-14) node[midway, above] {d};

%     % Draw the wires exiting the box with crossed mappings
%     \draw[-] (4-6, 2-14) -- (6-6,2-14) node[right, above] {a};
%     \draw[-] (4-6, 1-14) -- (6-6, 1-14) node[right, above] {b};
%     \draw[-] (4-6, 0-14) -- (6-6, 0-14) node[right, above] {c};
%     \draw[-] (4-6,-1-14) -- (6-6, -1-14) node[right, above] {d};

%     % Draw the lines inside the box to represent the mapping
%     \draw[-] (2-6, 2-14) -- (4-6,0-14);
%     \draw[-] (2-6, 1-14) -- (4-6, -1-14);
%     \draw[-] (2-6, 0-14) -- (4-6, 2-14);
%     \draw[-] (2-6,-1-14) -- (4-6, 1-14);

%     \node at (3-6, -2-14) {$\overline\delta_1$};

%     \draw[fill=lightgray] (2,-1.5-14) rectangle (4,2.5-14) node[midway] {};

%     \node at (3, -2-14) {$\overline\sigma_1 = \overline\delta_1$};

%     % Draw the wires entering the box
%     \draw[-] (0, 2-14) -- (2, 2-14) node[midway, above] {};
%     \draw[-] (0, 1-14) -- (2, 1-14) node[midway, above] {};
%     \draw[-] (0, 0-14) -- (2, 0-14) node[midway, above] {};
%     \draw[-] (0,-1-14) -- (2,-1-14) node[midway, above] {};

%     % Draw the wires exiting the box
%     \draw[-] (4, 2-14) -- (6,2-14) node[right, above] {a};
%     \draw[-] (4, 1-14) -- (6, 1-14) node[right, above] {b};
%     \draw[-] (4, 0-14) -- (6, 0-14) node[right, above] {c};
%     \draw[-] (4,-1-14) -- (6, -1-14) node[right, above] {d};

%     % Draw the lines inside the box to represent the mapping
%     \draw[-] (2, 2-14) -- (4, -1-14);
%     \draw[-] (2, 1-14) -- (4, 0-14);
%     \draw[-] (2, 0-14) -- (4, 1-14);
%     \draw[-] (2,-1-14) -- (4, 2-14);

%     \draw[fill=lightgray] (2+6,-1.5-14) rectangle (4+6,2.5-14)
% node[midway] {};

%     \node at (3, -2-14) {$\overline\sigma_1 = \overline\delta_1$};

%     % Draw the wires entering the box
%     \draw[-] (0+6, 2-14) -- (2+6, 2-14) node[midway, above] {};
%     \draw[-] (0+6, 1-14) -- (2+6, 1-14) node[midway, above] {};
%     \draw[-] (0+6, 0-14) -- (2+6, 0-14) node[midway, above] {};
%     \draw[-] (0+6,-1-14) -- (2+6,-1-14) node[midway, above] {};

%     % Draw the wires exiting the box
%     \draw[-] (4+6, 2-14) -- (6+6,2-14) node[right, above] {a};
%     \draw[-] (4+6, 1-14) -- (6+6, 1-14) node[right, above] {b};
%     \draw[-] (4+6, 0-14) -- (6+6, 0-14) node[right, above] {c};
%     \draw[-] (4+6,-1-14) -- (6+6, -1-14) node[right, above] {d};

%     % Draw the lines inside the box to represent the mapping
%     \draw[-] (2+6, 2-14) -- (4+6, -1-14);
%     \draw[-] (2+6, 1-14) -- (4+6, 0-14);
%     \draw[-] (2+6, 0-14) -- (4+6, 1-14);
%     \draw[-] (2+6,-1-14) -- (4+6, 2-14);

%     \node at (3+6, -2-14) {$\overline\delta_2$};
%     % \node[draw,circle] at (-4.5, -1) {$A$};
%     % \node[draw,circle] at (3, -9) {$B$};
%     % \node[draw,circle] at (10.5, -1) {$C$};
%   \end{tikzpicture}
% \end{center}


%\appendix
%\include{appendixA}

%%%%%%%%%%%%%%%%%%%%%%%%%%%%%%%%%%%%%%%%%%%%%%%%%%
%Backend of thesis (references and the like)
%%%%%%%%%%%%%%%%%%%%%%%%%%%%%%%%%%%%%%%%%%%%%%%%%%
% \backmatter

%Add your bibliography filename and uncomment these lines
% \bibliographystyle{amsplain}
% \addcontentsline{toc}{chapter}{References}
% \bibliography{citations}
% \markboth{{\sffamily\sc Bibliography}}{}

%Uncomment if you want an index
%\printindex

\end{document}
