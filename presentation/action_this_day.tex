\documentclass[hyphens,aspectratio=169]{beamer}
\usepackage{graphicx}
\usepackage{amssymb}
\usepackage{amsmath}
\usepackage{mathtools}
\usepackage{textcomp}
\usepackage{moresize}
\usepackage{framed}
\usepackage{minted}
\usepackage{relsize}

\usetheme{Berlin}
\usecolortheme[RGB={0,95,47}]{structure}
\beamertemplatenavigationsymbolsempty

\makeatletter
\AtBeginEnvironment{minted}{\dontdofcolorbox}
\def\dontdofcolorbox{\renewcommand\fcolorbox[4][]{##4}}
\setbeamertemplate{footline}{}            % removes the entire bottom bar
\makeatother
\title{Action This Day!}
\subtitle{The Mathematics and Machinations that Bested the German Enigma}
\author{Jonah Weinbaum}
\date{
July 28, 2025
}
\begin{document}

\frame{\titlepage}

\section{Action This Day}

\begin{frame}[fragile]{D-Day}
\end{frame}

\begin{frame}[fragile]{Letter to Winston Churchill}
\end{frame}

\begin{frame}[fragile]{Acknowledgments}
\end{frame}

\begin{frame}[fragile]{Thesis Overview}
\end{frame}

\part{}

\section{The Enigma}

\begin{frame}[fragile]{Slide}
\end{frame}

\subsection{Key Space}

\begin{frame}[fragile]{Slide}
\end{frame}

\part{}

\section{Group Theory and Permutations}

\begin{frame}[fragile]{Slide}
\end{frame}

\subsection{Enigma as a Permutation}

\begin{frame}[fragile]{Slide}
\end{frame}

\part{}
\section{The Cyclometer}

\subsection{Characteristics}
\begin{frame}[fragile]{Slide}
\end{frame}

\subsection{The Cyclometer}
\begin{frame}[fragile]{Slide}
\end{frame}

\part{}
\section{The Turing-Welchman Bombe}

\begin{frame}[fragile]{Slide}
\end{frame}

\part{}
\section{Extensions to The Bombe}

\begin{frame}[fragile]{Slide}
\end{frame}

\part{}
\section{Stops}

\begin{frame}[fragile]{Slide}
\end{frame}

\part{}
\section{The H-M Factor}

\subsection{Prior Work}

\begin{frame}[fragile]{Slide}
\end{frame}

\subsection{A New H-M Factor}

\begin{frame}[fragile]{Slide}
\end{frame}

\part{}
\section{Code Contributions}

\begin{frame}[fragile]{Slide}
\end{frame}

\part{}
\section{Future Work}

\begin{frame}[fragile]{Slide}
\end{frame}

\end{document}
